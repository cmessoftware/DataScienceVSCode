   \documentclass[handout]{beamer}



   \mode<presentation>
   {
     \usetheme{PaloAlto}
   \setbeamertemplate{footline}[page number]

     \setbeamercolor{sidebar}{bg=white, fg=black}
     \setbeamercolor{frametitle}{bg=white, fg=black}
     % or ...
     \setbeamercolor{logo}{bg=white}
     \setbeamercolor{block body}{parent=normal text,bg=white}
     \setbeamercolor{author in sidebar}{fg=black}
     \setbeamercolor{title in sidebar}{fg=black}


     \setbeamercolor*{block title}{use=structure,fg=structure.fg,bg=structure.fg!20!bg}
     \setbeamercolor*{block title alerted}{use=alerted text,fg=alerted text.fg,bg=alerted text.fg!20!bg}
     \setbeamercolor*{block title example}{use=example text,fg=example text.fg,bg=example text.fg!20!bg}


     \setbeamercolor{block body}{parent=normal text,use=block title,bg=block title.bg!50!bg}
     \setbeamercolor{block body alerted}{parent=normal text,use=block title alerted,bg=block title alerted.bg!50!bg}
     \setbeamercolor{block body example}{parent=normal text,use=block title example,bg=block title example.bg!50!bg}

     % or ...

     \setbeamercovered{transparent}
     % or whatever (possibly just delete it)
     \logo{\resizebox{!}{1.5cm}{\href{\basename{R}}{\includegraphics{image}}}}
   }

   \mode<handout>
   {
     \usetheme{PaloAlto}
     \usecolortheme{default}
     \setbeamercolor{sidebar}{bg=white, fg=black}
     \setbeamercolor{frametitle}{bg=white, fg=black}
     % or ...
     \setbeamercolor{logo}{bg=white}
     \setbeamercolor{block body}{parent=normal text,bg=white}
     \setbeamercolor{author in sidebar}{fg=black}
     \setbeamercolor{title in sidebar}{fg=black}
     \setbeamercovered{transparent}
     % or whatever (possibly just delete it)
     \logo{}
   }

   \usepackage{epsdice,listings,epic}
   \usepackage[latin1]{inputenc}
   \usepackage{graphicx}
   \usepackage{amsmath,eepic,epic,algorithm}

   \newcommand{\figslide}[3]{
   \begin{frame}
   \frametitle{#1}
     \begin{center}
     \resizebox{!}{2.7in}{\includegraphics{#2}}    
     \end{center}
   {#3}
   \end{frame}
   }

   \newcommand{\fighslide}[4]{
   \begin{frame}
   \frametitle{#1}
     \begin{center}
     \resizebox{!}{#4}{\includegraphics{#2}}    
     \end{center}
   {#3}
   \end{frame}
   }

   \newcommand{\figwref}[1]{
   \href{#1}{\tiny \tt #1}}

   \newcommand{\unsupervised}[1]{{\color{red} #1}}
   \newcommand{\supervised}[1]{{\color{green} #1}}
   \newcommand{\argmax}{\mathop{\mathrm{argmax}}}
   \newcommand{\argmin}{\mathop{\mathrm{argmin}}}
   \newcommand{\minimize}{\mathop{\mathrm{minimize}}}
   \newcommand{\maximize}{\mathop{\mathrm{maximize}}}

   \newcommand{\B}[1]{\beta_{#1}}
   \newcommand{\Bh}[1]{\widehat{\beta}_{#1}}
   \newcommand{\V}{\text{Var}}
   \newcommand{\Cov}{\text{Cov}}
   \newcommand{\Vh}{\widehat{\V}}
   \newcommand{\s}{\sigma}
   \newcommand{\sh}{\widehat{\sigma}}

   \newcommand{\argmax}[1]{\mathop{\text{argmax}}_{#1}}
   \newcommand{\argmin}[1]{\mathop{\text{argmin}}_{#1}}
   \newcommand{\Ee}{\mathbb{E}}
   \newcommand{\Pp}{\mathbb{P}}
   \newcommand{\real}{\mathbb{R}}
   \newcommand{\Ybar}{\overline{Y}}
   \newcommand{\Yh}{\widehat{Y}}
   \newcommand{\Xbar}{\overline{X}}
   \newcommand{\Tr}{\text{Tr}}


   \newcommand{\model}{{\cal M}}

   \newcommand{\figvskip}{-0.7in}
   \newcommand{\fighskip}{-0.3in}
   \newcommand{\figheight}{3.5in}

   \newcommand{\Rcode}[1]{{\bf \tt #1 }}
   \newcommand{\Rtcode}[1]{{\tiny \bf \tt #1 }}
   \newcommand{\Rscode}[1]{{\small \bf \tt #1 }}

   \newcommand{\RR}{{\tt R} \;}
   \newcommand{\basename}[1]{http://stats191.stanford.edu/#1}
   \newcommand{\dataname}[1]{\basename{data/#1}}
   \newcommand{\Rname}[1]{\basename{R/#1}}

   \newcommand{\mycolor}[1]{{\color{blue} #1}}
   \newcommand{\basehref}[2]{\href{\basename{#1}}{\mycolor{#2}}}
   \newcommand{\Rhref}[2]{\href{\basename{R/#1}}{\mycolor{#2}}}
   \newcommand{\datahref}[2]{\href{\dataname{#1}}{\mycolor{#2}}}
   \newcommand{\X}{\pmb{X}}
   \newcommand{\Y}{\pmb{Y}}
   \newcommand{\be}{\pmb{varepsilon}}
   \newcommand{\logit}{\text{logit}}


   \title{Statistics 191: Introduction to Applied Statistics}
   \subtitle{Bias-Variance tradeoff: penalized techniques (Ch. 10, RABE)}
   \author{\copyright Jonathan Taylor \\
   }
   %}


   \begin{document}

   \begin{frame}
   \titlepage
   \end{frame}

   %%%%%%%%%%%%%%%%%%%%%%%%%%%%%%%%%%%%%%%%%%%%%%%%%%%%%%%%%%%%

   \begin{frame} \frametitle{Penalized regression}

   \begin{block} {Topics}
     \begin{itemize}

     \item Bias-Variance tradeoff.
   \item Penalized regression.
     \item Cross-validation.

   \end{itemize}

   \end{block}
   \end{frame}

   %%%%%%%%%%%%%%%%%%%%%%%%%%%%%%%%%%%%%%%%%%%%%%%%%%%%%%%%%%%%

   \begin{frame} \frametitle{Penalized regression}

   \begin{block}
   {Bias-variance tradeoff                     }
       \begin{itemize}

       \item Arguably, the goal of a regression analysis
   is to ``build'' a model that predicts well.

   \item What does ``predict well'' mean?
   $$
   \begin{aligned}
   MSE_{pop}(\model) &= \Ee \left((Y_{new} - \widehat{Y}_{new,{\cal M}})^2\right) \\
   &=
   \V(Y_{\new}) + \V(\widehat{Y}_{new,{\cal M}}) +
   \\
   & \qquad \quad \text{Bias}(\widehat{Y}_{new,{\cal M}})^2.
   \end{aligned}
   $$
       \end{itemize}
   \end{block}
   \end{frame}

   %%%%%%%%%%%%%%%%%%%%%%%%%%%%%%%%%%%%%%%%%%%%%%%%%%%%%%%%%%%%

   \begin{frame} \frametitle{Bias-Variance Tradeoff}

   \begin{block}
   {Simulation}

   \begin{enumerate}
   \item Generate $Y_{10 \times 1} \sim N(\mu \pmb{1}, I_{10 \times 10}),
   with $\mu=0.5$.
   \item For $0 \leq \alpha \leq 1$, set
   $$
   \hat{Y}(\alpha) = \alpha \bar{Y}.
   $$
   \item Compute
   $$
   MSE(\hat{Y}(\alpha)) = \sum_{i=1}^{10} (\hat{Y}_{\alpha} - 0.5)^2
   $$
   \item Repeat 100 times, plot average of $MSE(\hat{Y}(\alpha))$.
   \end{itemize}
   \end{block}
   \end{frame}

   %CODE
       % nsample <- 10
   % ntrial <- 100
   % mu <- 0.5
   % 
   % MSE <- function(mu.hat, mu) {
   %   return(sum((mu.hat - mu)^2))
   % }
   % 
   % alpha <- seq(0.0,1,length=20)
   % 
   % mse <- numeric(length(alpha))
   % 
   % for (i in 1:ntrial) {
   %   Z <- rnorm(nsample) + mu
   %   for (j in 1:length(alpha)) {
   %     mse[j] <- mse[j] + MSE(alpha[j] * mean(Z), mu) / ntrial
   %   }
   % }
   % 
   % plot(alpha, mse, type='l', lwd=2, col='red', ylim=c(0, max(mse)),
   %      xlab='Shrinkage parameter, alpha', ylab='MSE(alpha)')


   \begin{frame}
   \frametitle{Bias-Variance Tradeoff}
   \begin{center}
   \resizebox{!}{2.7in}{\includegraphics{./images/inline/b48c9d4f9f.png}}    
   \end{center}
   \href{http://www.stanford.edu/class/stats191/selection.html#bias-variance-tradeoff}{R code}
   \end{frame}

   %%%%%%%%%%%%%%%%%%%%%%%%%%%%%%%%%%%%%%%%%%%%%%%%%%%%%%%%%%%%

   \begin{frame} \frametitle{Penalized regression}

   \begin{block}
     {Shrinkage \& Penalties}
     \begin{itemize}
     \item Shrinkage can be thought of as ``constrained'' minimization.

     \item Minimize
   $$
   \sum_{i=1}^n (Y_i - \mu)^2 \quad \text{subject to $\mu^2 \leq C$} $$

   \item Lagrange: equivalent to minimizing
   $$
   \sum_{i=1}^n (Y_i - \mu)^2 + \lambda \mu^2$$
   for some $\lambda=\lambda_C$
     \end{itemize}

   \end{block}
   \end{frame}

   %%%%%%%%%%%%%%%%%%%%%%%%%%%%%%%%%%%%%%%%%%%%%%%%%%%%%%%%%%%%

   \begin{frame} \frametitle{Penalized regression}

   \begin{block}
     {Shrinkage \& Penalties}
     \begin{itemize}
   \item Differentiating:
   $$- 2 \sum_{i=1}^n (Y_i - \widehat{\mu}_{\lambda}) + 2 \lambda \widehat{\mu}_{\lambda} = 0$$

   \item Solving
   $$
   \widehat{\mu}_{\lambda} = \frac{\sum_{i=1}^n Y_i}{n + \lambda} = \frac{n}{n+\lambda} \overline{Y}.$$
     \item As
   $\lambda \rightarrow 0$,
   $$
   \widehat{\mu}_{\lambda} \rightarrow \Ybar.$$

   \item As $\lambda \rightarrow \infty$
   $$
   \widehat{\mu}_{\lambda} \rightarrow 0.$$

     \end{itemize}

   \end{block}
   \end{frame}

   %%%%%%%%%%%%%%%%%%%%%%%%%%%%%%%%%%%%%%%%%%%%%%%%%%%%%%%%%%%%

   \begin{frame} \frametitle{Penalized regression}

   \begin{block}
     {Penalties \& Priors}
     \begin{itemize}
     \item Minimizing
   $$
   \sum_{i=1}^n (Y_i - \mu)^2 + \lambda \mu^2$$
   is similar to computing ``MLE'' of $\mu$ if the likelihood
   was proportional to
   $$
   \exp \left(-\frac{1}{2\sigma^2}\left( \sum_{i=1}^n (Y_i - \mu)^2 + \lambda \mu^2\right) \right).
   $$


   \item If $\lambda=m$, an integer, then $\widehat{\mu}_{\lambda}$ is the sample
   mean of $(Y_1, \dots, Y_n,0 ,\dots, 0) \in \mathbb{R}^{n+m}$.
   \item This is equivalent to adding some data with $Y=0$ \dots

     \end{itemize}

   \end{block}
   \end{frame}

   %%%%%%%%%%%%%%%%%%%%%%%%%%%%%%%%%%%%%%%%%%%%%%%%%%%%%%%%%%%%

   \begin{frame} \frametitle{Penalized regression}

   \begin{block}
   {Biased regression: penalties                     }
       \begin{itemize}[<+->]

       \item Not all biased models are better -- we need a way to find ``good'' biased models.

       \item Inference ($F$, $\chi^2$ tests, etc) is not quite exact for  biased models.

       \item Generalized one sample problem: penalize large values of $\B{}$.
       This should lead to ``multivariate'' shrinkage of the vector  $\B{}$.
       \item Heuristically, ``large $\B{}$'' is interpreted as ``complex model''. Goal is really to penalize ``complex'' models, i.e. Occam's razor.

       \item If truth  really is complex, this may not work! (But, it will then be hard to build a good model anyways ... (statistical lore))

       \end{itemize}

   \end{block}
   \end{frame}

   %%%%%%%%%%%%%%%%%%%%%%%%%%%%%%%%%%%%%%%%%%%%%%%%%%%%%%%%%%%%

   \begin{frame} \frametitle{Penalized regression}

   \begin{block}
         {How much to shrink, choosing $\alpha$}
         \begin{itemize}
         \item In our one-sample example,

   $$
   \begin{aligned}
   MSE_{pop}(\alpha) &= \V( \alpha \bar{Y}) + \text{Bias}(\alpha \bar{Y})^2 \\
   &= \frac{\alpha^2 \sigma^2}{n} + \mu^2 (1 - \alpha)^2
   \end{aligned}
   $$

   \item Differentiating and solving:
   $$
   \begin{aligned}
   0 &= -2 \mu^2(1 - \lambda^*) + 2 \frac{\lambda^* \sigma^2}{n}  \\
   \lambda^* & = \frac{\mu^2}{\mu^2+\sigma^2/n} \\
   &= \frac{0.5^2}{0.5^2+1/10} = 0.71
   \end{aligned}
   $$
         \end{itemize}
   \end{block}
   \end{frame}

   %CODE
       % nsample <- 10
   % ntrial <- 100
   % mu <- 0.5
   % 
   % 
   % MSE <- function(mu.hat, mu) {
   %   return(sum((mu.hat - mu)^2))
   % }
   % 
   % alpha <- seq(0.0,1,length=20)
   % 
   % mse <- numeric(length(alpha))
   % 
   % for (i in 1:ntrial) {
   %   Z <- rnorm(nsample) + mu
   %   for (j in 1:length(alpha)) {
   %     mse[j] <- mse[j] + MSE(alpha[j] * mean(Z), mu) / ntrial
   %   }
   % }
   % 
   % plot(alpha, mse, type='l', lwd=2, col='red', ylim=c(0, max(mse)),
   %      xlab='Shrinkage parameter, alpha', ylab='MSE(alpha)')
   % 
   % abline(v=mu^2/(mu^2+1/nsample), col='blue', lty=2)


   \begin{frame}
   \frametitle{Bias-Variance Tradeoff}
   \begin{center}
   \resizebox{!}{2.7in}{\includegraphics{./images/inline/cfa9a2a964.png}}    
   \end{center}
   \href{http://www.stanford.edu/class/stats191/selection.html#bias-variance-tradeoff}{R code}
   \end{frame}

   %%%%%%%%%%%%%%%%%%%%%%%%%%%%%%%%%%%%%%%%%%%%%%%%%%%%%%%%%%%%

   \begin{frame} \frametitle{Penalized regression}

   \begin{block}
   {Ridge regression}
       \begin{itemize}

       \item Assume that columns $(X_j)_{1 \leq j \leq p}$ have zero mean, and length 1 and $Y$ has zero mean.

       \item This is called the {\em standardized model}.
       \item Minimize
   {\small
   $$
   SSE_{\lambda}(\B{}) = \sum_{i=1}^n \left(Y_i - \sum_{j=1}^{p} X_{ij} \B{j}\right)^2 + \lambda \sum_{j=1}^{p} \B{j}^2.$$}

   \item Corresponds (through Lagrange multiplier) to a quadratic constraint on $\B{}$'s.

       \end{itemize}

   \end{block}
   \end{frame}

   %%%%%%%%%%%%%%%%%%%%%%%%%%%%%%%%%%%%%%%%%%%%%%%%%%%%%%%%%%%%

   \begin{frame} \frametitle{Penalized regression}

   \begin{block}
       {Solving the normal equations}
       \begin{itemize}
   \item Normal equations
   $$
   \frac{\partial}{\partial \B{l}} SSE_{\lambda}(\B{}) = -2  \langle Y - X\B{}, X_l \rangle + 2 \lambda \B{l}$$

       \item $$
   -2 \langle Y - X\Bh{\lambda}, X_l \rangle + 2 \lambda \Bh{l,\lambda} = 0, \qquad 1 \leq l \leq p$$

   \item In matrix form
   $$
   -X^TY +  (X^TX + \lambda I) \Bh{\lambda} = 0$$

   \item Or
   $$
   \Bh{\lambda} = (X^TX + \lambda I)^{-1} X^TY.$$
       \end{itemize}

   \end{block}
   \end{frame}

   %CODE
       % library(lars)
   % data(diabetes)
   % 
   % library(MASS)
   % 
   % diabetes.ridge <- lm.ridge(diabetes$y ~ diabetes$x, lambda=seq(0,10,0.05))
   % 
   % plot(diabetes.ridge)


   \begin{frame}
   \frametitle{Ridge regression}
   \begin{center}
   \resizebox{!}{2.7in}{\includegraphics{./images/inline/410687e961.png}}    
   \end{center}
   \href{http://www.stanford.edu/class/stats191/penalized.html#ridge-regression}{R code}
   \end{frame}

   %%%%%%%%%%%%%%%%%%%%%%%%%%%%%%%%%%%%%%%%%%%%%%%%%%%%%%%%%%%%

   \begin{frame} \frametitle{Penalized regression}

   \begin{block}
   {Choosing $\lambda$: cross-validation  }
       \begin{itemize}

       \item If we knew $MSE$ as a function of $\lambda$ then we would simply choose the $\lambda$ that minimizes $MSE$.

       \item To do this, we need to estimate $MSE$.

       \item A popular method is cross-validation as a function
       of $\lambda$. Breaks the data up into smaller groups and uses part of the data to predict the rest.

       \item We saw this in diagnostics (Cook's distance measured the fit with and without each point in the data set) and model selection.
       \end{itemize}

   \end{block}
   \end{frame}

   %%%%%%%%%%%%%%%%%%%%%%%%%%%%%%%%%%%%%%%%%%%%%%%%%%%%%%%%%%%%

   \begin{frame} \frametitle{Penalized regression}

   \begin{block}
   {$K$-fold cross-validation                     }
       \begin{itemize}

       \item Fix a model (i.e. fix $\lambda$). Break data set into $K$ approximately equal sized groups $(G_1, \dots, G_K)$.

       \item {\tt for (i in 1:K)} Use all groups except $G_i$ to fit model, predict  outcome in group $G_i$ based on this model $\widehat{Y}_{j(i),\lambda}, j \in G_i$.

       \item Estimate
   $$
   CV(\lambda) = \frac{1}{n}\sum_{i=1}^K \sum_{j \in G_i} (Y_j - \widehat{Y}_{j(i),\lambda})^2.$$
       \end{itemize}

   \end{block}
   \end{frame}

   %CODE
       %  ntrial = 500
   %  mu = 0.08
   %  nsample = 200
   % 
   %  MSE = function(mu.hat, mu) {
   %      return(sum((mu.hat - mu)^2))
   %  }
   % 
   % 
   %  # K fold cross-validation
   % 
   %  K = 10
   %  CV = function(Z, alpha) {
   %      cve = numeric(K)
   %      l = length(Z)
   %      for (i in 1:K) {
   %          g = c((i-1)*l/K+1,i*l/K)
   %          mu.hat = mean(Z[-g]) * alpha
   %          cve[i] = sum((Z[g]-mu.hat)^2)
   %      }
   %      return(c(mean(cve), sd(cve)))
   %  }
   % 
   %  alpha = seq(0.0,1,length=20)
   % 
   %  mse = numeric(length(alpha))
   %  cv = numeric(length(alpha))
   %  cv.sd = numeric(length(alpha))
   % 
   %  for (i in 1:ntrial) {
   %      Z = rnorm(nsample) + mu
   %      for (j in 1:length(alpha)) {
   %          mse[j] = mse[j] + MSE(alpha[j] * mean(Z), mu) / ntrial
   %          d = CV(Z, alpha[j])
   %          cv[j] = cv[j] + d[1] / ntrial
   %          cv.sd[j] = cv.sd[j] + d[2] / ntrial
   %      }
   %  }
   % 
   %  plot(alpha, cv, type='l', lwd=2, col='green',
   %     xlab='Shrinkage parameter, alpha', ylab='CV(alpha)')
   %  abline(v=mu^2/(mu^2+1/nsample), col='blue', lty=2)
   %  abline(v=min(alpha[cv == min(cv)]), col='red', lty=2)


   \begin{frame}
   \frametitle{Penalized regression}
   \begin{center}
   \resizebox{!}{2.7in}{\includegraphics{./images/inline/cfe8b24a70.png}}    
   \end{center}
   \href{http://www.stanford.edu/class/stats191/selection.html#bias-variance-tradeoff}{R code}
   \end{frame}

   %%%%%%%%%%%%%%%%%%%%%%%%%%%%%%%%%%%%%%%%%%%%%%%%%%%%%%%%%%%%

   \begin{frame} \frametitle{Penalized regression}

   \begin{block}
   {$K$-fold cross-validation   (continued)                  }
       \begin{itemize}

       \item It is a general principle that can be used in other situations, not just for Ridge.


       \item Pros (partial list): ``objective'' measure of a model.

       \item Cons (partial list): inference is, strictly speaking, ``out the window'' (also true for other model selection procedures in theory).

       \item If goal is not really inference about certain specific parameters, it is a reasonable way to compare models.
       \end{itemize}

   \end{block}
   \end{frame}

   %%%%%%%%%%%%%%%%%%%%%%%%%%%%%%%%%%%%%%%%%%%%%%%%%%%%%%%%%%%%

   \begin{frame} \frametitle{Penalized regression}

   \begin{block}
   { Generalized Cross Validation}
       \begin{itemize}

       \item A computational shortcut for $n$-fold cross-validation (also known as leave-one out cross-validation).

       \item Let
   $$
   S_{\lambda} = (X^TX + \lambda I)^{-1} X^T$$
   be the matrix in ridge regression.

   \item Then
   $$
   GCV(\lambda) =  \frac{\|Y - S_{\lambda}Y\|^2}{n - \Tr(S_{\lambda})}.$$

   \item The quantity $\Tr(S_{\lambda})$ is the {\em effective degrees of freedom}.
       \end{itemize}

   \end{block}
   \end{frame}

   %CODE
       % plot(diabetes.ridge$lambda, diabetes.ridge$GCV, xlab='Lambda', ylab='GCV', type='l', lwd=3, col='orange')
   % 
   % 
   % select(diabetes.ridge)


   \begin{frame}
   \frametitle{Ridge regression}
   \begin{center}
   \resizebox{!}{2.7in}{\includegraphics{./images/inline/6fbda6e3b3.png}}    
   \end{center}
   \href{http://www.stanford.edu/class/stats191/penalized.html#ridge-regression}{R code}
   \end{frame}

   %%%%%%%%%%%%%%%%%%%%%%%%%%%%%%%%%%%%%%%%%%%%%%%%%%%%%%%%%%%%

   \begin{frame} \frametitle{Penalized regression}

   \begin{block}
    {LASSO regression}
        \begin{itemize}
        \item Another popular penalized regression technique.

        \item Use the standardized model.
        \item Minimize
    $$
    SSE_{\lambda}(\B{}) = \sum_{i=1}^n \left(Y_i - \sum_{j=1}^{p} X_{ij} \B{j}\right)^2 + \lambda \sum_{j=1}^p |\B{j}|.$$

    \item Corresponds (through Lagrange multiplier) to an $\ell^1$ constraint on $\B{}$'s. In theory, it works well when many $\B{j}$'s are 0 and gives ``sparse'' solutions unlike ridge.

        \end{itemize}

   \end{block}
   \end{frame}

   %%%%%%%%%%%%%%%%%%%%%%%%%%%%%%%%%%%%%%%%%%%%%%%%%%%%%%%%%%%%

   \begin{frame} \frametitle{Penalized regression}

     \begin{figure}
         \centering
         \resizebox{3.5in}{!}{\includegraphics{figs/penalized/lasso_balls}}

         Why do we get sparse solutions with the LASSO?

        \end{figure}
   \end{frame}

   %CODE
       % library(lars)
   % data(diabetes)
   % 
   % diabetes.lasso = lars(diabetes$x, diabetes$y, type='lasso')
   % plot(diabetes.lasso)


   \begin{frame}
   \frametitle{LASSO}
   \begin{center}
   \resizebox{!}{2.7in}{\includegraphics{./images/inline/5405c7af69.png}}    
   \end{center}
   \href{http://www.stanford.edu/class/stats191/penalized.html#lasso}{R code}
   \end{frame}

   %%%%%%%%%%%%%%%%%%%%%%%%%%%%%%%%%%%%%%%%%%%%%%%%%%%%%%%%%%%%

   \begin{frame} \frametitle{Penalized regression}

   \begin{block}
       {Solving the normal equations}
       \begin{itemize}
   \item Normal equations
   $$
   \frac{\partial}{\partial \B{l}} SSE_{\lambda}(\B{}) = -2  \langle Y - X\B{}, X_l \rangle +  \lambda \frac{\partial}{\partial\B{l}} \|\B{}\|_1$$

       \item Problem: the $\ell_1$ norm is not differentiable.


   \item Normal equations are replaced by KKT (Karush-Kuhn-Tucker) conditions.
   $$
   -X^TY +  (X^TX + \lambda I) \Bh{\lambda} = \lambda \cdot s(\Bh{\lambda}).$$
   where $s(\Bh{\lambda})$ is a {\em subgradient} of the $\ell_1$ norm at
   $\Bh{\lambda}$ \dots
       \end{itemize}

   \end{block}
   \end{frame}

   %%%%%%%%%%%%%%%%%%%%%%%%%%%%%%%%%%%%%%%%%%%%%%%%%%%%%%%%%%%%

   \begin{frame} \frametitle{Penalized regression}

   \begin{block}
    {LASSO regression}
        \begin{itemize}
        \item Equivalent to the dual problem:

        \item Minimize
    $$
     \sum_{i=1}^n (Y_i - r_i)^2  \ \text{subject to} \ |X_j^Tr| \leq \lambda, 1 \leq j \leq p.$$

     \item So, instead of finding closest point to a plane, it finds closest
     (residual) point to a polygon...

        \end{itemize}

   \end{block}
   \end{frame}

   %%%%%%%%%%%%%%%%%%%%%%%%%%%%%%%%%%%%%%%%%%%%%%%%%%%%%%%%%%%%

   \begin{frame} \frametitle{Penalized regression}

   \begin{block}
    {LASSO regression}
        \begin{itemize}
        \item Scales to huge data because for
        large $\lambda$ solutions are {\em sparse}.

        \item If there is a good model that is sparse, then LASSO can usually
        find a good model.
        \end{itemize}

   \end{block}
   \end{frame}

   %CODE
       % cv.lars(diabetes$x, diabetes$y, K=10, type='lasso')


   \begin{frame}
   \frametitle{LASSO: cross-validation}
   \begin{center}
   \resizebox{!}{2.7in}{\includegraphics{./images/inline/28ed713a9a.png}}    
   \end{center}
   \href{http://www.stanford.edu/class/stats191/penalized.html#lasso}{R code}
   \end{frame}

   %%%%%%%%%%%%%%%%%%%%%%%%%%%%%%%%%%%%%%%%%%%%%%%%%%%%%%%%%%%%

   \begin{frame} \frametitle{Penalized regression}

   \begin{block}
    {Elastic Net}
        \begin{itemize}
        \item Mix between LASSO and ridge regression.

        \item Minimize
    $$
    SSE_{\lambda}(\B{}) = \sum_{i=1}^n \left(Y_i - \sum_{j=1}^{p-1} X_{ij} \B{j}\right)^2 + \lambda_1 \|\B\|_1 + \lambda_2 \|\B\|_2^2.$$

        \end{itemize}

   \end{block}
   \end{frame}

   %CODE
       % library(lars)
   % library(elasticnet)
   % data(diabetes)
   % 
   % diabetes.enet = enet(diabetes$x, diabetes$y, 0.1)
   % plot(diabetes.enet)


   \begin{frame}
   \frametitle{Elastic Net}
   \begin{center}
   \resizebox{!}{2.7in}{\includegraphics{./images/inline/ed9553f075.png}}    
   \end{center}
   \href{http://www.stanford.edu/class/stats191/penalized.html#lasso}{R code}
   \end{frame}

   %CODE
       % library(elasticnet)
   % data(diabetes)
   % 
   % diabetes.enet = enet(diabetes$x, diabetes$y, 10)
   % plot(diabetes.enet)


   \begin{frame}
   \frametitle{Elastic Net}
   \begin{center}
   \resizebox{!}{2.7in}{\includegraphics{./images/inline/371f9f664f.png}}    
   \end{center}
   \href{http://www.stanford.edu/class/stats191/penalized.html#lasso}{R code}
   \end{frame}

   %%%%%%%%%%%%%%%%%%%%%%%%%%%%%%%%%%%%%%%%%%%%%%%%%%%%%%%%%%%%

   \begin{frame} 

   \end{frame}

   \end{document}
