   \documentclass[handout]{beamer}



   \mode<presentation>
   {
     \usetheme{PaloAlto}
   \setbeamertemplate{footline}[page number]

     \setbeamercolor{sidebar}{bg=white, fg=black}
     \setbeamercolor{frametitle}{bg=white, fg=black}
     % or ...
     \setbeamercolor{logo}{bg=white}
     \setbeamercolor{block body}{parent=normal text,bg=white}
     \setbeamercolor{author in sidebar}{fg=black}
     \setbeamercolor{title in sidebar}{fg=black}


     \setbeamercolor*{block title}{use=structure,fg=structure.fg,bg=structure.fg!20!bg}
     \setbeamercolor*{block title alerted}{use=alerted text,fg=alerted text.fg,bg=alerted text.fg!20!bg}
     \setbeamercolor*{block title example}{use=example text,fg=example text.fg,bg=example text.fg!20!bg}


     \setbeamercolor{block body}{parent=normal text,use=block title,bg=block title.bg!50!bg}
     \setbeamercolor{block body alerted}{parent=normal text,use=block title alerted,bg=block title alerted.bg!50!bg}
     \setbeamercolor{block body example}{parent=normal text,use=block title example,bg=block title example.bg!50!bg}

     % or ...

     \setbeamercovered{transparent}
     % or whatever (possibly just delete it)
     \logo{\resizebox{!}{1.5cm}{\href{\basename{R}}{\includegraphics{image}}}}
   }

   \mode<handout>
   {
     \usetheme{PaloAlto}
     \usecolortheme{default}
     \setbeamercolor{sidebar}{bg=white, fg=black}
     \setbeamercolor{frametitle}{bg=white, fg=black}
     % or ...
     \setbeamercolor{logo}{bg=white}
     \setbeamercolor{block body}{parent=normal text,bg=white}
     \setbeamercolor{author in sidebar}{fg=black}
     \setbeamercolor{title in sidebar}{fg=black}
     \setbeamercovered{transparent}
     % or whatever (possibly just delete it)
     \logo{}
   }

   \usepackage{epsdice,listings,epic}
   \usepackage[latin1]{inputenc}
   \usepackage{graphicx}
   \usepackage{amsmath,eepic,epic,algorithm}

   \newcommand{\figslide}[3]{
   \begin{frame}
   \frametitle{#1}
     \begin{center}
     \resizebox{!}{2.7in}{\includegraphics{#2}}    
     \end{center}
   {#3}
   \end{frame}
   }

   \newcommand{\fighslide}[4]{
   \begin{frame}
   \frametitle{#1}
     \begin{center}
     \resizebox{!}{#4}{\includegraphics{#2}}    
     \end{center}
   {#3}
   \end{frame}
   }

   \newcommand{\figwref}[1]{
   \href{#1}{\tiny \tt #1}}

   \newcommand{\unsupervised}[1]{{\color{red} #1}}
   \newcommand{\supervised}[1]{{\color{green} #1}}
   \newcommand{\argmax}{\mathop{\mathrm{argmax}}}
   \newcommand{\argmin}{\mathop{\mathrm{argmin}}}
   \newcommand{\minimize}{\mathop{\mathrm{minimize}}}
   \newcommand{\maximize}{\mathop{\mathrm{maximize}}}

   \newcommand{\B}[1]{\beta_{#1}}
   \newcommand{\Bh}[1]{\widehat{\beta}_{#1}}
   \newcommand{\V}{\text{Var}}
   \newcommand{\Cov}{\text{Cov}}
   \newcommand{\Vh}{\widehat{\V}}
   \newcommand{\s}{\sigma}
   \newcommand{\sh}{\widehat{\sigma}}

   \newcommand{\argmax}[1]{\mathop{\text{argmax}}_{#1}}
   \newcommand{\argmin}[1]{\mathop{\text{argmin}}_{#1}}
   \newcommand{\Ee}{\mathbb{E}}
   \newcommand{\Pp}{\mathbb{P}}
   \newcommand{\real}{\mathbb{R}}
   \newcommand{\Ybar}{\overline{Y}}
   \newcommand{\Yh}{\widehat{Y}}
   \newcommand{\Xbar}{\overline{X}}
   \newcommand{\Tr}{\text{Tr}}


   \newcommand{\model}{{\cal M}}

   \newcommand{\figvskip}{-0.7in}
   \newcommand{\fighskip}{-0.3in}
   \newcommand{\figheight}{3.5in}

   \newcommand{\Rcode}[1]{{\bf \tt #1 }}
   \newcommand{\Rtcode}[1]{{\tiny \bf \tt #1 }}
   \newcommand{\Rscode}[1]{{\small \bf \tt #1 }}

   \newcommand{\RR}{{\tt R} \;}
   \newcommand{\basename}[1]{http://stats191.stanford.edu/#1}
   \newcommand{\dataname}[1]{\basename{data/#1}}
   \newcommand{\Rname}[1]{\basename{R/#1}}

   \newcommand{\mycolor}[1]{{\color{blue} #1}}
   \newcommand{\basehref}[2]{\href{\basename{#1}}{\mycolor{#2}}}
   \newcommand{\Rhref}[2]{\href{\basename{R/#1}}{\mycolor{#2}}}
   \newcommand{\datahref}[2]{\href{\dataname{#1}}{\mycolor{#2}}}
   \newcommand{\X}{\pmb{X}}
   \newcommand{\Y}{\pmb{Y}}
   \newcommand{\be}{\pmb{varepsilon}}
   \newcommand{\logit}{\text{logit}}


   \title{Statistics 191: Introduction to Applied Statistics}
   \subtitle{Poisson regression}
   \author{\copyright Jonathan Taylor \\
   }
   %}


   \begin{document}

   \begin{frame}
   \titlepage
   \end{frame}

   %%%%%%%%%%%%%%%%%%%%%%%%%%%%%%%%%%%%%%%%%%%%%%%%%%%%%%%%%%%%

   \begin{frame} \frametitle{Poisson regression}

   \begin{block}
   {Topics}
   \begin{itemize}


     \item Contingency tables.

     \item Poisson regression.


     \item Generalized linear model.

   \end{itemize}

   \end{block}
   \end{frame}

   %%%%%%%%%%%%%%%%%%%%%%%%%%%%%%%%%%%%%%%%%%%%%%%%%%%%%%%%%%%%

   \begin{frame} \frametitle{Count data}

   \begin{block}
     {Afterlife}
     \begin{itemize}
     \item   Men and women were asked whether they believed in the
   after life (1991 General Social Survey).

   \item
     \begin{tabular}{l|c|c|c}
       & Y & N or U & Total \\ \hline
   M & 435 & 147 & 582 \\
   F & 375 & 134 & 509 \\ \hline
   Total & 810 & 281 & 1091
     \end{tabular}

   \item Question: is belief in the afterlife independent of gender?

     \end{itemize}
   \end{block}
   \end{frame}

   %%%%%%%%%%%%%%%%%%%%%%%%%%%%%%%%%%%%%%%%%%%%%%%%%%%%%%%%%%%%

   \begin{frame} \frametitle{Poisson counts}

   \begin{block}
   {Definition}
   \begin{itemize}
   \item A random variable $Y$ is a Poisson random variable
   with parameter $\lambda$ if
   $$
   P(Y=j) = e^{-\lambda} \frac{\lambda^j}{j!}, \qquad \forall j \geq 0.
   $$
   \item Some simple calculations show that
   $$
   E(Y)=\text{Var}(Y)=\lambda.
   $$
   \item Poisson models for counts are analogous to Gaussian for continuous
   outcomes.
   \end{itemize}
   \end{block}
   \end{frame}

   %%%%%%%%%%%%%%%%%%%%%%%%%%%%%%%%%%%%%%%%%%%%%%%%%%%%%%%%%%%%

   \begin{frame} \frametitle{Count data}

   \begin{block}
       {Contingency table}

       \begin{itemize}
       \item     Model: $Y_{ij} \sim  Poisson(\lambda_{ij} )$.


       \item {\bf Null:}
       $H_0 : \text{independence}, \lambda_{ij} = \lambda \alpha_i \cdot \beta_j , \sum_i \alpha_i = 1,  \sum_j \beta_j = 1.$

     \item {\bf
       Alternative:} $H_a : \text{$\lambda_{ij}$ 's are unrestricted}$

   \item {\bf    Test statistic:} Pearson's $X^2$ :
   $$
   X^2 = \sum_{ij} \frac{(Y_{ij}-E_{ij})^2}{E_{ij}} \approx \chi^2_1 \  \text{under $H_0$}$$

   \item
       Why 1 df ? Independence model has 5 parameters, two
       constraints = 3 df. Unrestricted has 4 parameters.

     \item This is actually a {\em regression model} for the count data.

       \end{itemize}

   \end{block}
   \end{frame}

   %%%%%%%%%%%%%%%%%%%%%%%%%%%%%%%%%%%%%%%%%%%%%%%%%%%%%%%%%%%%

   \begin{frame} \frametitle{Count data}

   \begin{block}
   {Contingency table as regression model}
   \begin{itemize}
   \item     Under independence
   $$
   \begin{aligned}
     \log(E (Y_{ij} )) &= \log \lambda_{ij} = \log \lambda  + \log \alpha_i + \log \beta_j
   \end{aligned}
   $$

   \item     OR, the model has a {\em log link}.

   \item     What about the variance? Because of Poisson assumption
          $$                  Var(Y_{ij} ) = E (Y_{ij})$$

        \item     OR, the {\em variance function} is
                             $$V (\mu) = \mu.$$
   \end{itemize}
   \end{block}
   \end{frame}

   %%%%%%%%%%%%%%%%%%%%%%%%%%%%%%%%%%%%%%%%%%%%%%%%%%%%%%%%%%%%

   \begin{frame} \frametitle{Count data}

   \begin{block}
       {Contingency table $(k \times m)$}
       \begin{itemize}
       \item     Suppose we had $k$ categories on one axis, $m$ on the other
       (i.e. previous example $k = m = 2$). We call this as $k \times m$
       contingency table.

     \item Independence model:
   $$        \log(E (Y_{ij} )) = \log \lambda_{ij} = \log \lambda  + \log \alpha_i + \log \beta_j$$



       \end{itemize}
   \end{block}
   \end{frame}

   %%%%%%%%%%%%%%%%%%%%%%%%%%%%%%%%%%%%%%%%%%%%%%%%%%%%%%%%%%%%

   \begin{frame} \frametitle{Count data}

   \begin{block}
       {Contingency tables}
       \begin{itemize}
    \item Test for independence: Pearson's
   $$
   X^2 = \sum_{ij} \frac{(Y_{ij}-E_{ij})^2}{E_{ij}} \approx \chi^2_{(k-1)(m-1)} \  \text{under $H_0$}$$


       \item Alternative test statistic
   $$
   G = 2\sum_{ij} Y_{ij} \log \left(\frac{Y_{ij}}{E_{ij}}\right)$$
       \end{itemize}
   \end{block}
   \end{frame}

   %%%%%%%%%%%%%%%%%%%%%%%%%%%%%%%%%%%%%%%%%%%%%%%%%%%%%%%%%%%%

   \begin{frame} \frametitle{Count data}

   \begin{block}
       {Independence tests}
       \begin{itemize}[<+->]
       \item Unlike in other cases, in this case the {\em full model}
       has as many parameters as observations (i.e. it's saturated).

       \item This test is known as a {\em goodness of fit} test.
       \item {\em How well does the independence model fit this data}?
       \end{itemize}
   \end{block}
   \end{frame}

   %%%%%%%%%%%%%%%%%%%%%%%%%%%%%%%%%%%%%%%%%%%%%%%%%%%%%%%%%%%%

   \begin{frame} \frametitle{Count data}

   \begin{block}
       {Lumber company example}
       \begin{itemize}
       \item $Y$ : number of customers visting store from region;

    \item $X_1$ : number of housing units in region;

    \item $X_2$ : average household income;

    \item $X_3$ : average housing unit age in region;

    \item $X_4$ : distance to nearest competitor;

    \item $X_5$ : distance to store in miles.

       \end{itemize}
   \end{block}
   \end{frame}

   %%%%%%%%%%%%%%%%%%%%%%%%%%%%%%%%%%%%%%%%%%%%%%%%%%%%%%%%%%%%

   \begin{frame} \frametitle{Count data}

   \begin{block}
       {Poisson (log-linear) regression model}

       \begin{itemize}
       \item      Given observations and covariates
        $Y_i , X_{ij} , 1 \leq i  \leq n, 1 \leq j  \leq p$.


      \item {\bf Model:}
   $$     Y_{i} \sim Poisson \left(\exp\left(\beta_0 + \sum_{j=1}^p \beta_j X_{ij} \right)\right)$$


   \item Poisson assumption implies the  variance
        function is
   $$ V (\mu) = \mu.$$
       \end{itemize}

   \end{block}
   \end{frame}

   %%%%%%%%%%%%%%%%%%%%%%%%%%%%%%%%%%%%%%%%%%%%%%%%%%%%%%%%%%%%

   \begin{frame} \frametitle{Poisson regression}

   \begin{block}
   {Interpretation of coefficients}

   \begin{itemize}
   \item   The log-linear model
   means covariates have {\em multiplicative} effect.

   \item Log-linear model model:
   $$
   \frac{E(Y|\dots, X_j=x_j+1, \dots)}{E(Y|\dots, X_j=x_j, \dots)} = e^{\beta_j}
   $$


   \item So, one unit increase in variable $j$ results in $e^{\beta_j}$
   (multiplicative) increase the expected count, all other parameters being equal.

   \end{itemize}
   \end{block}
   \end{frame}

   %%%%%%%%%%%%%%%%%%%%%%%%%%%%%%%%%%%%%%%%%%%%%%%%%%%%%%%%%%%%

   \begin{frame} \frametitle{Count data}

   \begin{block}
         {Generalized linear models}
         \begin{itemize}
         \item Logistic model:
   $$ \logit(\pi) = \beta_0 + \sum_j \beta_j X_j \qquad V(\pi)=\pi(1-\pi)$$

   \item Poisson log-linear model:
   $$
   \log(\mu) = \beta_0 + \sum_j \beta_j X_j, \qquad V(\mu) = \mu$$

   \item These are the ingredients to a GLM \dots
         \end{itemize}
   \end{block}
   \end{frame}

   %%%%%%%%%%%%%%%%%%%%%%%%%%%%%%%%%%%%%%%%%%%%%%%%%%%%%%%%%%%%

   \begin{frame} \frametitle{Generalized linear models}

   \begin{block}
         {Specifying a model}
         \begin{itemize}
         \item Given $(Y, X_1, \dots, X_p)$, a GLM is specified by the
   (link, variance function) pair $(V, g)$.

   \item Fit using IRLS like logistic.

   \item Inference in terms of deviance or Pearson's $X^2$:
   $$
   X^2({\cal M}) = \sum_{i=1}^n \frac{(Y_i - \widehat{\mu}_{{\cal M},i})^2}{V(\widehat{\mu}_{{\cal M},i})}$$
         \end{itemize}
   \end{block}
   \end{frame}

   %%%%%%%%%%%%%%%%%%%%%%%%%%%%%%%%%%%%%%%%%%%%%%%%%%%%%%%%%%%%

   \begin{frame} \frametitle{Generalized linear models}

   \begin{block}
         {Deviance}
         \begin{itemize}
         \item Replaces $SSE$ in least squares

         \item Definition
   $$
   DEV({\cal M}) = -2 \left(\log L(\widehat{\mu}({\cal M})|Y,X) - \log(Y|Y,X) \right) $$

   \item Difference between fitted values of ${\cal M}$ and "saturated model" with $\widehat{\mu}=Y$.


   \item Poisson deviance
   $$
   DEV({\cal M}|Y) = 2 \sum_{i=1}^n \left( Y_i \log \left(Y_i / \widehat{\mu}_{{\cal M},i} \right) + (Y_i - \widehat{\mu}_{{\cal M},i} ) \right)
   $$
         \end{itemize}
   \end{block}
   \end{frame}

   %%%%%%%%%%%%%%%%%%%%%%%%%%%%%%%%%%%%%%%%%%%%%%%%%%%%%%%%%%%%

   \begin{frame} \frametitle{Generalized linear models}

   \begin{block}
         {Deviance tests}
         \begin{itemize}
         \item To test $H_0:{\cal M}={\cal M}_R$ vs. $H_a: {\cal M}={\cal M}_F$,
   we use
   $$DEV({\cal M}_R) - DEV({\cal M}_F) \sim \chi^2_{df_R-df_F}$$

   \item In contingency example ${\cal M}_R$ is the independence model
   $$
   \log(E(Y_{ij})) = \lambda + \alpha_i  + \beta_j$$
   with ${\cal M}_F$ being the ``saturated model.''


         \end{itemize}
   \end{block}
   \end{frame}

   %%%%%%%%%%%%%%%%%%%%%%%%%%%%%%%%%%%%%%%%%%%%%%%%%%%%%%%%%%%%

   \begin{frame} 

   \end{frame}

   \end{document}
