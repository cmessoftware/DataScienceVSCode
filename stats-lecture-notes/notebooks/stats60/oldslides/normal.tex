   \documentclass[handout]{beamer}



   \mode<presentation>
   {
     \usetheme{PaloAlto}
   \setbeamertemplate{footline}[page number]

     \setbeamercolor{sidebar}{bg=white, fg=black}
     \setbeamercolor{frametitle}{bg=white, fg=black}
     % or ...
     \setbeamercolor{logo}{bg=white}
     \setbeamercolor{block body}{parent=normal text,bg=white}
     \setbeamercolor{author in sidebar}{fg=black}
     \setbeamercolor{title in sidebar}{fg=black}


     \setbeamercolor*{block title}{use=structure,fg=structure.fg,bg=structure.fg!20!bg}
     \setbeamercolor*{block title alerted}{use=alerted text,fg=alerted text.fg,bg=alerted text.fg!20!bg}
     \setbeamercolor*{block title example}{use=example text,fg=example text.fg,bg=example text.fg!20!bg}


     \setbeamercolor{block body}{parent=normal text,use=block title,bg=block title.bg!50!bg}
     \setbeamercolor{block body alerted}{parent=normal text,use=block title alerted,bg=block title alerted.bg!50!bg}
     \setbeamercolor{block body example}{parent=normal text,use=block title example,bg=block title example.bg!50!bg}

     % or ...

     \setbeamercovered{transparent}
     % or whatever (possibly just delete it)
     \logo{\resizebox{!}{1.5cm}{\href{\basename{R}}{\includegraphics{image}}}}
   }

   \mode<handout>
   {
     \usetheme{PaloAlto}
     \usecolortheme{default}
     \setbeamercolor{sidebar}{bg=white, fg=black}
     \setbeamercolor{frametitle}{bg=white, fg=black}
     % or ...
     \setbeamercolor{logo}{bg=white}
     \setbeamercolor{block body}{parent=normal text,bg=white}
     \setbeamercolor{author in sidebar}{fg=black}
     \setbeamercolor{title in sidebar}{fg=black}
     \setbeamercovered{transparent}
     % or whatever (possibly just delete it)
     \logo{}
   }

   \usepackage{epsdice}
   \usepackage[latin1]{inputenc}
   \usepackage{graphics}
   \usepackage{amsmath,eepic,epic}

   \newcommand{\figslide}[3]{
   \begin{frame}
   \frametitle{#1}
     \begin{center}
     \resizebox{!}{2.7in}{\includegraphics{#2}}    
     \end{center}
   {#3}
   \end{frame}
   }

   \newcommand{\fighslide}[4]{
   \begin{frame}
   \frametitle{#1}
     \begin{center}
     \resizebox{!}{#4}{\includegraphics{#2}}    
     \end{center}
   {#3}
   \end{frame}
   }

   \newcommand{\figwref}[1]{
   \href{#1}{\tiny \tt #1}}

   \newcommand{\B}[1]{\beta_{#1}}
   \newcommand{\Bh}[1]{\widehat{\beta}_{#1}}
   \newcommand{\V}{\text{Var}}
   \newcommand{\Cov}{\text{Cov}}
   \newcommand{\Vh}{\widehat{\V}}
   \newcommand{\s}{\sigma}
   \newcommand{\sh}{\widehat{\sigma}}

   \newcommand{\argmax}[1]{\mathop{\text{argmax}}_{#1}}
   \newcommand{\argmin}[1]{\mathop{\text{argmin}}_{#1}}
   \newcommand{\Ee}{\mathbb{E}}
   \newcommand{\Pp}{\mathbb{P}}
   \newcommand{\real}{\mathbb{R}}
   \newcommand{\Ybar}{\overline{Y}}
   \newcommand{\Yh}{\widehat{Y}}
   \newcommand{\Xbar}{\overline{X}}
   \newcommand{\Tr}{\text{Tr}}


   \newcommand{\model}{{\cal M}}

   \newcommand{\figvskip}{-0.7in}
   \newcommand{\fighskip}{-0.3in}
   \newcommand{\figheight}{3.5in}

   \newcommand{\Rcode}[1]{{\bf \tt #1 }}
   \newcommand{\Rtcode}[1]{{\tiny \bf \tt #1 }}
   \newcommand{\Rscode}[1]{{\small \bf \tt #1 }}

   \newcommand{\RR}{{\tt R} \;}
   \newcommand{\basename}[1]{http://stats60.stanford.edu/#1}
   \newcommand{\dataname}[1]{\basename{data/#1}}
   \newcommand{\Rname}[1]{\basename{R/#1}}

   \newcommand{\mycolor}[1]{{\color{blue} #1}}
   \newcommand{\basehref}[2]{\href{\basename{#1}}{\mycolor{#2}}}
   \newcommand{\Rhref}[2]{\href{\basename{R/#1}}{\mycolor{#2}}}
   \newcommand{\datahref}[2]{\href{\dataname{#1}}{\mycolor{#2}}}
   \newcommand{\X}{\pmb{X}}
   \newcommand{\Y}{\pmb{Y}}
   \newcommand{\be}{\pmb{varepsilon}}
   \newcommand{\logit}{\text{logit}}


   \title{Statistics 60: Introduction to Statistical Methods}
   \subtitle{Chapter 5: Normal Approximation for Data} 
   \author{}% {Jonathan Taylor \\
   %Department of Statistics \\
   %Stanford University
   %}


   \begin{document}

   \begin{frame}
   \titlepage
   \end{frame}

   %%%%%%%%%%%%%%%%%%%%%%%%%%%%%%%%%%%%%%%%%%%%%%%%%%%%%%%%%%%%

   \begin{frame} \frametitle{Standardized units}

   \begin{block}
   {Standardizing a list of numbers}
   \begin{itemize}
   \item Subtracting the average of a list of numbers from each
   entry yields a new list with average 0.
   \item Dividing each entry in a list of numbers by its SD
   yields a new list with SD 1.
   \item Combining the two operations yields a {\em standardized} list,
   where entries are in {\em standard units}.
   \item Standard units describe how many SDs each each is above or below the
   average of the list.
   \end{itemize}
   \end{block}
   \end{frame}

   %%%%%%%%%%%%%%%%%%%%%%%%%%%%%%%%%%%%%%%%%%%%%%%%%%%%%%%%%%%%

   \begin{frame} \frametitle{Standardized units}

   \begin{block}
   {Rule for standardizing {\em list}}
   $$
   \text{new entry ({\em list})} = \frac{\text{old entry({\em list})} - \text{average({\em list})}}{\text{SD({\em list})}}
   $$
   \end{block}

   \begin{block}
   {Example: [161 \text{lbs}, 166 \text{lbs}, 171 \text{lbs}, 172 \text{lbs}]}


   The average is 167.5 \text{lbs} and the  SD is $\approx 4.4 \text{lbs}$
   $$
   \begin{aligned}
   \text{standardized list} &= [(161-167.5)/4.4,   (166-167.5)/4.4, \\
   & \qquad (171-167.5)/4.4, (172-167.5)/4.4] \\
   &= [-1.5,-0.3,0.8,1.0]
   \end{aligned}
   $$

   \end{block}
   \end{frame}

   %%%%%%%%%%%%%%%%%%%%%%%%%%%%%%%%%%%%%%%%%%%%%%%%%%%%%%%%%%%%

   \begin{frame} \frametitle{Standardized units}

   \begin{block}
   {$\Sigma$ notation}
   \begin{itemize}
   \item If our list is $X=[X_1, \dots, X_n]$ and we call our
   standardized list $Z=[Z_1, \dots, Z_n]$.

   \item Then,
   $$
   Z_i = \frac{X_i - \bar{X}}{\text{SD}(X)}.
   $$
   \end{itemize}
   \end{block}
   \end{frame}

   %CODE
       % from matplotlib import rc
   % import pylab, numpy as np, sys
   % np.random.seed(0);import random; random.seed(0)
   % sys.path.append('/private/var/folders/dq/4_9bwd013ln6vvf_q110mwrh0000gn/T/tmpVgHiPk')
   % f=pylab.gcf(); f.set_size_inches(8.0,6.0)
   % datadir ='/private/var/folders/dq/4_9bwd013ln6vvf_q110mwrh0000gn/T/tmpVgHiPk/data'
   % import pylab, numpy as np
   % x = np.linspace(-4,4,101)
   % y = np.exp(-x**2/2) / np.sqrt(2*np.pi)
   % pylab.plot(x,y*100, linewidth=2)
   % pylab.gca().set_xlabel('standardized units')
   % pylab.gca().set_ylabel('% per standardized unit')
   % #pylab.gca().set_xlim([-2,4])
   % #pylab.gca().set_yticks([])
   % 


   \begin{frame}
   \frametitle{The normal curve}
   \begin{center}
   \resizebox{!}{2.7in}{\includegraphics{./images/inline/5514bb9f41.pdf}}    
   \end{center}
   Many histograms follow the normal curve after standardizing
   \end{frame}

   %%%%%%%%%%%%%%%%%%%%%%%%%%%%%%%%%%%%%%%%%%%%%%%%%%%%%%%%%%%%

   \begin{frame} \frametitle{The normal curve}

   \begin{block}
   {Rules of thumb for SD}
   \begin{itemize}
   \item The area under the normal curve between -1 and +1 is about $68\%$.
   \item The area under the normal curve between -2 and +2 is about $95\%$.
   \item The area under the normal curve between -3 and +3 is about $99.7\%$.
   \end{itemize}
   \end{block}
   \end{frame}

   %%%%%%%%%%%%%%%%%%%%%%%%%%%%%%%%%%%%%%%%%%%%%%%%%%%%%%%%%%%%

   \begin{frame} \frametitle{The normal table: A-104}

   \begin{block}
   {Sample rows of the table look like}
   \begin{tabular}{rrr}
   $z$ & Height & Area \\ \hline
   0.70 & 31.23 & 51.61 \\
   1.00 & 24.20 & 68.27 \\
   2.00 & 5.40 & 95.45 \\
   \end{tabular}
   \end{block}
   \end{frame}

   %CODE
       % from matplotlib import rc
   % import pylab, numpy as np, sys
   % np.random.seed(0);import random; random.seed(0)
   % sys.path.append('/private/var/folders/dq/4_9bwd013ln6vvf_q110mwrh0000gn/T/tmpVgHiPk')
   % f=pylab.gcf(); f.set_size_inches(8.0,6.0)
   % datadir ='/private/var/folders/dq/4_9bwd013ln6vvf_q110mwrh0000gn/T/tmpVgHiPk/data'
   % import pylab, numpy as np
   % x = np.linspace(-4,4,101)
   % y = np.exp(-x**2/2) / np.sqrt(2*np.pi)
   % pylab.bar([-0.75,0.65], [31.23,31.23], width=0.1, color='red')
   % pylab.plot(x,y*100, linewidth=2)
   % pylab.gca().set_xlabel('standardized units')
   % pylab.gca().set_ylabel('% per standardized unit')
   % #pylab.gca().set_xlim([-2,4])
   % #pylab.gca().set_yticks([])
   % 


   \begin{frame}
   \frametitle{Using table A-104}
   \begin{center}
   \resizebox{!}{2.7in}{\includegraphics{./images/inline/5a118e8784.pdf}}    
   \end{center}
   The height at  $z= \pm 0.7$ is $31.23\%$ (per standardized unit)
   \end{frame}

   %CODE
       % from matplotlib import rc
   % import pylab, numpy as np, sys
   % np.random.seed(0);import random; random.seed(0)
   % sys.path.append('/private/var/folders/dq/4_9bwd013ln6vvf_q110mwrh0000gn/T/tmpVgHiPk')
   % f=pylab.gcf(); f.set_size_inches(8.0,6.0)
   % datadir ='/private/var/folders/dq/4_9bwd013ln6vvf_q110mwrh0000gn/T/tmpVgHiPk/data'
   % import pylab, numpy as np
   % x = np.linspace(-4,4,101)
   % y = np.exp(-x**2/2) / np.sqrt(2*np.pi)
   % x2 = np.linspace(-0.7,0.7,101)
   % y2 = np.exp(-x2**2/2) / np.sqrt(2*np.pi)
   % pylab.plot(x,y*100, linewidth=2)
   % xf, yf = pylab.poly_between(x2, 0*x2, y2*100)
   % pylab.fill(xf, yf, facecolor='r', hatch='\\', alpha=0.5)
   % pylab.gca().set_xlabel('standardized units')
   % pylab.gca().set_ylabel('% per standardized unit')
   % #pylab.gca().set_xlim([-2,4])
   % #pylab.gca().set_yticks([])
   % 


   \begin{frame}
   \frametitle{Using table A-104}
   \begin{center}
   \resizebox{!}{2.7in}{\includegraphics{./images/inline/92f14628eb.pdf}}    
   \end{center}
   The area between $z=-0.7$ and $z=0.7$ is $51.61\%$
   \end{frame}

   %CODE
       % from matplotlib import rc
   % import pylab, numpy as np, sys
   % np.random.seed(0);import random; random.seed(0)
   % sys.path.append('/private/var/folders/dq/4_9bwd013ln6vvf_q110mwrh0000gn/T/tmpVgHiPk')
   % f=pylab.gcf(); f.set_size_inches(8.0,6.0)
   % datadir ='/private/var/folders/dq/4_9bwd013ln6vvf_q110mwrh0000gn/T/tmpVgHiPk/data'
   % import pylab, numpy as np
   % x = np.linspace(-4,4,101)
   % y = np.exp(-x**2/2) / np.sqrt(2*np.pi)
   % x2 = np.linspace(0,0.7,101)
   % y2 = np.exp(-x2**2/2) / np.sqrt(2*np.pi)
   % pylab.plot(x,y*100, linewidth=2)
   % xf, yf = pylab.poly_between(x2, 0*x2, y2*100)
   % pylab.fill(xf, yf, facecolor='r', hatch='\\', alpha=0.5)
   % pylab.gca().set_xlabel('standardized units')
   % pylab.gca().set_ylabel('% per standardized unit')
   % #pylab.gca().set_xlim([-2,4])
   % #pylab.gca().set_yticks([])
   % 


   \begin{frame}
   \frametitle{Computing area under the normal curve}
   \begin{center}
   \resizebox{!}{2.7in}{\includegraphics{./images/inline/853133f328.pdf}}    
   \end{center}
   Area = ????
   \end{frame}

   %CODE
       % from matplotlib import rc
   % import pylab, numpy as np, sys
   % np.random.seed(0);import random; random.seed(0)
   % sys.path.append('/private/var/folders/dq/4_9bwd013ln6vvf_q110mwrh0000gn/T/tmpVgHiPk')
   % f=pylab.gcf(); f.set_size_inches(8.0,6.0)
   % datadir ='/private/var/folders/dq/4_9bwd013ln6vvf_q110mwrh0000gn/T/tmpVgHiPk/data'
   % import pylab, numpy as np
   % x = np.linspace(-4,4,101)
   % y = np.exp(-x**2/2) / np.sqrt(2*np.pi)
   % x2 = np.array([0.7] + list(np.linspace(0.7,4,101)) + [4])
   % y2 = np.exp(-x2**2/2) / np.sqrt(2*np.pi)
   % y2[0] = 0; y2[-1] = 0
   % pylab.plot(x,y*100, linewidth=2)
   % xf, yf = pylab.poly_between(x2, 0 * x2, y2*100)
   % pylab.fill(xf, yf, facecolor='r', hatch='\\', alpha=0.5)
   % pylab.gca().set_xlabel('standardized units')
   % pylab.gca().set_ylabel('% per standardized unit')
   % #pylab.gca().set_xlim([-2,4])
   % #pylab.gca().set_yticks([])
   % 


   \begin{frame}
   \frametitle{Computing area under the normal curve}
   \begin{center}
   \resizebox{!}{2.7in}{\includegraphics{./images/inline/4c4c4b6d12.pdf}}    
   \end{center}
   Area = ????
   \end{frame}

   %CODE
       % from matplotlib import rc
   % import pylab, numpy as np, sys
   % np.random.seed(0);import random; random.seed(0)
   % sys.path.append('/private/var/folders/dq/4_9bwd013ln6vvf_q110mwrh0000gn/T/tmpVgHiPk')
   % f=pylab.gcf(); f.set_size_inches(8.0,6.0)
   % datadir ='/private/var/folders/dq/4_9bwd013ln6vvf_q110mwrh0000gn/T/tmpVgHiPk/data'
   % import pylab, numpy as np
   % x = np.linspace(-4,4,101)
   % y = np.exp(-x**2/2) / np.sqrt(2*np.pi)
   % x2 = np.linspace(0.7,4,101)
   % y2 = np.exp(-x2**2/2) / np.sqrt(2*np.pi)
   % pylab.plot(x,y*100, linewidth=2)
   % xf, yf = pylab.poly_between(x2, 0*x2, y2*100)
   % xf2, yf2 = pylab.poly_between(-x2, 0*x2, y2*100)
   % pylab.fill(xf, yf, facecolor='r', hatch='\\', alpha=0.5)
   % pylab.fill(xf2, yf2, facecolor='r', hatch='\\', alpha=0.5)
   % pylab.gca().set_xlabel('standardized units')
   % pylab.gca().set_ylabel('% per standardized unit')
   % #pylab.gca().set_xlim([-2,4])
   % #pylab.gca().set_yticks([])
   % 


   \begin{frame}
   \frametitle{Computing area under the normal curve}
   \begin{center}
   \resizebox{!}{2.7in}{\includegraphics{./images/inline/cad0bd54d9.pdf}}    
   \end{center}
   Area = ????
   \end{frame}

   %CODE
       % from matplotlib import rc
   % import pylab, numpy as np, sys
   % np.random.seed(0);import random; random.seed(0)
   % sys.path.append('/private/var/folders/dq/4_9bwd013ln6vvf_q110mwrh0000gn/T/tmpVgHiPk')
   % f=pylab.gcf(); f.set_size_inches(8.0,6.0)
   % datadir ='/private/var/folders/dq/4_9bwd013ln6vvf_q110mwrh0000gn/T/tmpVgHiPk/data'
   % import pylab, numpy as np
   % x = np.linspace(-4,4,101)
   % y = np.exp(-x**2/2) / np.sqrt(2*np.pi)
   % x2 = np.linspace(-4,0.7,101)
   % y2 = np.exp(-x2**2/2) / np.sqrt(2*np.pi)
   % pylab.plot(x,y*100, linewidth=2)
   % xf, yf = pylab.poly_between(x2, 0*x2, y2*100)
   % pylab.fill(xf, yf, facecolor='r', hatch='\\', alpha=0.5)
   % pylab.gca().set_xlabel('standardized units')
   % pylab.gca().set_ylabel('% per standardized unit')
   % #pylab.gca().set_xlim([-2,4])
   % #pylab.gca().set_yticks([])
   % 


   \begin{frame}
   \frametitle{Computing area under the normal curve}
   \begin{center}
   \resizebox{!}{2.7in}{\includegraphics{./images/inline/9e35d8424c.pdf}}    
   \end{center}
   Area = ????
   \end{frame}

   %CODE
       % from matplotlib import rc
   % import pylab, numpy as np, sys
   % np.random.seed(0);import random; random.seed(0)
   % sys.path.append('/private/var/folders/dq/4_9bwd013ln6vvf_q110mwrh0000gn/T/tmpVgHiPk')
   % f=pylab.gcf(); f.set_size_inches(8.0,6.0)
   % datadir ='/private/var/folders/dq/4_9bwd013ln6vvf_q110mwrh0000gn/T/tmpVgHiPk/data'
   % import pylab, numpy as np
   % x = np.linspace(-4,4,101)
   % y = np.exp(-x**2/2) / np.sqrt(2*np.pi)
   % x2 = np.linspace(-2,1,101)
   % y2 = np.exp(-x2**2/2) / np.sqrt(2*np.pi)
   % pylab.plot(x,y*100, linewidth=2)
   % xf, yf = pylab.poly_between(x2, 0*x2, y2*100)
   % pylab.fill(xf, yf, facecolor='r', hatch='\\', alpha=0.5)
   % pylab.gca().set_xlabel('standardized units')
   % pylab.gca().set_ylabel('% per standardized unit')
   % #pylab.gca().set_xlim([-2,4])
   % #pylab.gca().set_yticks([])
   % 


   \begin{frame}
   \frametitle{Computing area under the normal curve}
   \begin{center}
   \resizebox{!}{2.7in}{\includegraphics{./images/inline/7f04f35afa.pdf}}    
   \end{center}
   Area = ????
   \end{frame}

   %CODE
       % from matplotlib import rc
   % import pylab, numpy as np, sys
   % np.random.seed(0);import random; random.seed(0)
   % sys.path.append('/private/var/folders/dq/4_9bwd013ln6vvf_q110mwrh0000gn/T/tmpVgHiPk')
   % f=pylab.gcf(); f.set_size_inches(8.0,6.0)
   % datadir ='/private/var/folders/dq/4_9bwd013ln6vvf_q110mwrh0000gn/T/tmpVgHiPk/data'
   % import pylab, numpy as np
   % x = np.linspace(-4,4,101)
   % y = np.exp(-x**2/2) / np.sqrt(2*np.pi)
   % x2 = np.linspace(-2,0,101)
   % y2 = np.exp(-x2**2/2) / np.sqrt(2*np.pi)
   % pylab.plot(x,y*100, linewidth=2)
   % xf, yf = pylab.poly_between(x2, 0*x2, y2*100)
   % pylab.fill(xf, yf, facecolor='green', hatch='\\', alpha=0.5)
   % x2 = np.linspace(0,1,101)
   % y2 = np.exp(-x2**2/2) / np.sqrt(2*np.pi)
   % xf, yf = pylab.poly_between(x2, 0*x2, y2*100)
   % pylab.fill(xf, yf, facecolor='blue', hatch='o', alpha=0.5)
   % pylab.gca().set_xlabel('standardized units')
   % pylab.gca().set_ylabel('% per standardized unit')
   % #pylab.gca().set_xlim([-2,4])
   % #pylab.gca().set_yticks([])
   % 


   \begin{frame}
   \frametitle{Computing area under the normal curve}
   \begin{center}
   \resizebox{!}{2.7in}{\includegraphics{./images/inline/d78adcdcb5.pdf}}    
   \end{center}
   Can be solved by breaking the region up into two \dots
   \end{frame}

   %%%%%%%%%%%%%%%%%%%%%%%%%%%%%%%%%%%%%%%%%%%%%%%%%%%%%%%%%%%%

   \begin{frame} \frametitle{Normal approximation for data}

   \begin{block}
   {Example}
   The heights of the male freshmen at Stanford averaged 68 inches
   with an SD of 3 inches. Use the normal curve to estimate the
   percentage of these men between 62 inches and 71 inches.
   \end{block}
   \end{frame}

   %CODE
       % from matplotlib import rc
   % import pylab, numpy as np, sys
   % np.random.seed(0);import random; random.seed(0)
   % sys.path.append('/private/var/folders/dq/4_9bwd013ln6vvf_q110mwrh0000gn/T/tmpVgHiPk')
   % f=pylab.gcf(); f.set_size_inches(8.0,6.0)
   % datadir ='/private/var/folders/dq/4_9bwd013ln6vvf_q110mwrh0000gn/T/tmpVgHiPk/data'
   % xf,yf = pylab.poly_between([62,68,71],[-0.1,-0.1,-0.1],[0,0,0])
   % pylab.fill(xf,yf, hatch='/', alpha=0.5)
   % pylab.gca().set_ylim([-1,1])
   % pylab.gca().set_xlim([60,72])
   % pylab.gca().set_xticks([62,71])
   % pylab.gca().set_xlabel('Inches')
   % pylab.gca().set_yticks([])
   % 


   \begin{frame}
   \frametitle{Normal approximation for data}
   \begin{center}
   \resizebox{!}{2.7in}{\includegraphics{./images/inline/73f1c7ccb7.pdf}}    
   \end{center}
   Step 1: draw the interval
   \end{frame}

   %CODE
       % from matplotlib import rc
   % import pylab, numpy as np, sys
   % np.random.seed(0);import random; random.seed(0)
   % sys.path.append('/private/var/folders/dq/4_9bwd013ln6vvf_q110mwrh0000gn/T/tmpVgHiPk')
   % f=pylab.gcf(); f.set_size_inches(8.0,6.0)
   % datadir ='/private/var/folders/dq/4_9bwd013ln6vvf_q110mwrh0000gn/T/tmpVgHiPk/data'
   % xf,yf = pylab.poly_between([62,68,71],[-0.1,-0.1,-0.1],[0,0,0])
   % pylab.fill(xf,yf, hatch='/', alpha=0.5)
   % pylab.plot([68,68], [-0.1,0], linewidth=3, c='r')
   % pylab.gca().set_ylim([-1,1])
   % pylab.gca().set_xlim([60,72])
   % pylab.gca().set_xticks([62,68,71])
   % pylab.gca().set_xlabel('Inches')
   % pylab.gca().set_yticks([])
   % 


   \begin{frame}
   \frametitle{Normal approximation for data}
   \begin{center}
   \resizebox{!}{2.7in}{\includegraphics{./images/inline/211f84180a.pdf}}    
   \end{center}
   Step 2: add the average to the interval
   \end{frame}

   %CODE
       % from matplotlib import rc
   % import pylab, numpy as np, sys
   % np.random.seed(0);import random; random.seed(0)
   % sys.path.append('/private/var/folders/dq/4_9bwd013ln6vvf_q110mwrh0000gn/T/tmpVgHiPk')
   % f=pylab.gcf(); f.set_size_inches(8.0,6.0)
   % datadir ='/private/var/folders/dq/4_9bwd013ln6vvf_q110mwrh0000gn/T/tmpVgHiPk/data'
   % xf,yf = pylab.poly_between([-2,0,1],[-0.1,-0.1,-0.1],[0,0,0])
   % pylab.fill(xf,yf, hatch='/', alpha=0.5)
   % pylab.plot([0,0], [-0.1,0], linewidth=3, c='r')
   % pylab.gca().set_ylim([-1,1])
   % pylab.gca().set_xlim([(60-68)/3.,(72-68)/3.])
   % pylab.gca().set_xticks([-2,0,1])
   % pylab.gca().set_xlabel('Standardized units')
   % pylab.gca().set_yticks([])
   % 


   \begin{frame}
   \frametitle{Normal approximation for data}
   \begin{center}
   \resizebox{!}{2.7in}{\includegraphics{./images/inline/849bf5a4f4.pdf}}    
   \end{center}
   Step 3: convert to standardized units
   \end{frame}

   %CODE
       % from matplotlib import rc
   % import pylab, numpy as np, sys
   % np.random.seed(0);import random; random.seed(0)
   % sys.path.append('/private/var/folders/dq/4_9bwd013ln6vvf_q110mwrh0000gn/T/tmpVgHiPk')
   % f=pylab.gcf(); f.set_size_inches(8.0,6.0)
   % datadir ='/private/var/folders/dq/4_9bwd013ln6vvf_q110mwrh0000gn/T/tmpVgHiPk/data'
   % import pylab, numpy as np
   % x = np.linspace(-4,4,101)
   % y = np.exp(-x**2/2) / np.sqrt(2*np.pi)
   % x2 = np.linspace(-2,0,101)
   % y2 = np.exp(-x2**2/2) / np.sqrt(2*np.pi)
   % pylab.plot(x,y*100, linewidth=2)
   % xf, yf = pylab.poly_between(x2, 0*x2, y2*100)
   % pylab.fill(xf, yf, facecolor='green', hatch='\\', alpha=0.5)
   % x2 = np.linspace(0,1,101)
   % y2 = np.exp(-x2**2/2) / np.sqrt(2*np.pi)
   % xf, yf = pylab.poly_between(x2, 0*x2, y2*100)
   % pylab.fill(xf, yf, facecolor='blue', hatch='o', alpha=0.5)
   % pylab.gca().set_xlabel('standardized units')
   % pylab.gca().set_ylabel('% per standardized unit')
   % #pylab.gca().set_xlim([-2,4])
   % #pylab.gca().set_yticks([])
   % 


   \begin{frame}
   \frametitle{Computing area under the normal curve}
   \begin{center}
   \resizebox{!}{2.7in}{\includegraphics{./images/inline/d78adcdcb5.pdf}}    
   \end{center}
   Can be solved by breaking the region up into two \dots
   \end{frame}

   %CODE
       % from matplotlib import rc
   % import pylab, numpy as np, sys
   % np.random.seed(0);import random; random.seed(0)
   % sys.path.append('/private/var/folders/dq/4_9bwd013ln6vvf_q110mwrh0000gn/T/tmpVgHiPk')
   % f=pylab.gcf(); f.set_size_inches(8.0,6.0)
   % datadir ='/private/var/folders/dq/4_9bwd013ln6vvf_q110mwrh0000gn/T/tmpVgHiPk/data'
   % import pylab, numpy as np
   % x = np.linspace(-4,4,101)
   % y = np.exp(-x**2/2) / np.sqrt(2*np.pi)
   % x2 = np.linspace(-2,0,101)
   % y2 = np.exp(-x2**2/2) / np.sqrt(2*np.pi)
   % pylab.plot(x,y*100, linewidth=2)
   % xf, yf = pylab.poly_between(x2, 0*x2, y2*100)
   % pylab.fill(xf, yf, facecolor='green', hatch='\\', alpha=0.5)
   % x2 = np.linspace(0,1,101)
   % y2 = np.exp(-x2**2/2) / np.sqrt(2*np.pi)
   % xf, yf = pylab.poly_between(x2, 0*x2, y2*100)
   % pylab.fill(xf, yf, facecolor='blue', hatch='o', alpha=0.5)
   % pylab.gca().set_xlabel('standardized units')
   % pylab.gca().set_ylabel('% per standardized unit')
   % #pylab.gca().set_xlim([-2,4])
   % #pylab.gca().set_yticks([])
   % 


   \begin{frame}
   \frametitle{Computing area under the normal curve}
   \begin{center}
   \resizebox{!}{2.7in}{\includegraphics{./images/inline/d78adcdcb5.pdf}}    
   \end{center}
   Area = {\color{green}{95.45 \% / 2} }+ {\color{blue}{68.27 \% / 2}} $\approx$ 82 \%
   \end{frame}

   %CODE
       % from matplotlib import rc
   % import pylab, numpy as np, sys
   % np.random.seed(0);import random; random.seed(0)
   % sys.path.append('/private/var/folders/dq/4_9bwd013ln6vvf_q110mwrh0000gn/T/tmpVgHiPk')
   % f=pylab.gcf(); f.set_size_inches(8.0,6.0)
   % datadir ='/private/var/folders/dq/4_9bwd013ln6vvf_q110mwrh0000gn/T/tmpVgHiPk/data'
   % import matplotlib.mlab as ML
   % A = ML.csv2rec('%s/householdincome2006.csv' % datadir, delimiter=';')
   % from denshist import denshist, patchopt
   % breaks = np.array([0, 10,20,30,40,50,60,70,80,90,100,150,200,np.inf])*1000
   % count = []
   % for l, u in zip(breaks, breaks[1:]):
   %     g = (A['lower'] < u) * (A['upper'] >= l)
   %     count.append(A['count'][g].sum())
   % breaks[-1] = 300000
   % denshist(count, breaks, **patchopt)
   % #denshist(A['count'], list(A['lower']) + [300000], **patchopt)
   % pylab.gca().set_yticks([])
   % pylab.gca().set_xticks(np.arange(13)*25000)
   % pylab.gca().set_xticklabels(np.arange(13)*25)
   % pylab.gca().set_xlabel('Income (1000s)')
   % 


   \begin{frame}
   \frametitle{Sometimes the normal approximation doesn't work}
   \begin{center}
   \resizebox{!}{2.7in}{\includegraphics{./images/inline/d833554974.pdf}}    
   \end{center}
   http://pubdb3.census.gov/macro/032007/hhinc/new06_000.htm
   \end{frame}

   %%%%%%%%%%%%%%%%%%%%%%%%%%%%%%%%%%%%%%%%%%%%%%%%%%%%%%%%%%%%

   \begin{frame} \frametitle{Percentiles}

   \begin{block}
     {Definition}
     \begin{itemize}
     \item The 1st percentile of a histogram is the number with 1 \% of the
     area to the left and 99 \% of the area to the right.

     \item The 10th percentile of a histogram is the number with 10 \% of the
     area to the left and 90 \% of the area to the right.

     \item The median is the 50th percentie.

     \item The {\em first quartile} is the 25th percentile; the {\em third quartile} is the 75th percentile.

     \item The {\em inter-quartile range} is the difference between
     the third and first quartiles.
     \end{itemize}
   \end{block}
   \end{frame}

   %%%%%%%%%%%%%%%%%%%%%%%%%%%%%%%%%%%%%%%%%%%%%%%%%%%%%%%%%%%%

   \begin{frame} \frametitle{Percentiles}

   \begin{block}
   {Example: income distribution}

   \begin{tabular}{rr}
   Percent & Income percentile \\
   2 & \$2,500 \\
   25 & \$25,000 \\
   50 & \$ 49,000 \\
   75 & \$ 85,000 \\
   90 & \$ 135,000 \\
   99 & \$ 250,000
   \end{tabular}

   The interquartile range is \$60,000. The median is \$49,000.

   \end{block}
   \end{frame}

   %CODE
       % from matplotlib import rc
   % import pylab, numpy as np, sys
   % np.random.seed(0);import random; random.seed(0)
   % sys.path.append('/private/var/folders/dq/4_9bwd013ln6vvf_q110mwrh0000gn/T/tmpVgHiPk')
   % f=pylab.gcf(); f.set_size_inches(8.0,6.0)
   % datadir ='/private/var/folders/dq/4_9bwd013ln6vvf_q110mwrh0000gn/T/tmpVgHiPk/data'
   % import pylab, numpy as np
   % x = np.linspace(-4,4,101)
   % y = np.exp(-x**2/2) / np.sqrt(2*np.pi)
   % x2 = np.linspace(-4,1.65,101)
   % y2 = np.exp(-x2**2/2) / np.sqrt(2*np.pi)
   % pylab.plot(x,y*100, linewidth=2)
   % xf, yf = pylab.poly_between(x2, 0*x2, y2*100)
   % pylab.fill(xf, yf, facecolor='red', hatch='\\', alpha=0.5)
   % pylab.gca().set_xlabel('standardized units')
   % pylab.gca().set_ylabel('% per standardized unit')
   % #pylab.gca().set_xlim([-2,4])
   % #pylab.gca().set_yticks([])
   % 


   \begin{frame}
   \frametitle{Percentiles: normal curve}
   \begin{center}
   \resizebox{!}{2.7in}{\includegraphics{./images/inline/c1973fe581.pdf}}    
   \end{center}
   95\%-ile is 1.65 (or, {\color{red} red area is 95 \%})
   \end{frame}

   %%%%%%%%%%%%%%%%%%%%%%%%%%%%%%%%%%%%%%%%%%%%%%%%%%%%%%%%%%%%

   \begin{frame} \frametitle{Percentiles}

   \begin{block}
   {Example: normal curve}

   \begin{tabular}{rr}
   Percent & Normal percentile \\
   2.5 & -1.96 \\
   5 &  -1.65 \\
   25 & -0.68 \\
   50 & 0 \\
   75 & 0.68 \\
   90 & 1.28 \\
   95 & 1.65 \\
   97.5 & 1.96 \\
   \end{tabular}

   The interquartile range is 1.34. The median is 0.

   \end{block}
   \end{frame}

   %%%%%%%%%%%%%%%%%%%%%%%%%%%%%%%%%%%%%%%%%%%%%%%%%%%%%%%%%%%%

   \begin{frame} \frametitle{Percentiles}

   \begin{block}
   {Example: estimating percentiles}

   Among all applicants to Stanford the Math SAT scores averaged 560 with
   an SD of 100. The scores followed the normal distribution
   quite well. Estimate the 90th percentile of the score distribution.

   \end{block}

   \begin{block}
     {Solution}

   The estimated quantile is
   $$
   \begin{aligned}
   \lefteqn{560 + 100 \times \text{90 \%-ile of the normal curve}} \\
   & \qquad =  560 + 100 \times 1.28 \\
   & \qquad = 688
   \end{aligned}
   $$
   \end{block}
   \end{frame}

   %%%%%%%%%%%%%%%%%%%%%%%%%%%%%%%%%%%%%%%%%%%%%%%%%%%%%%%%%%%%

   \begin{frame} 

   \end{frame}

   \end{document}
