   \documentclass[handout]{beamer}



   \mode<presentation>
   {
     \usetheme{PaloAlto}
   \setbeamertemplate{footline}[page number]

     \setbeamercolor{sidebar}{bg=white, fg=black}
     \setbeamercolor{frametitle}{bg=white, fg=black}
     % or ...
     \setbeamercolor{logo}{bg=white}
     \setbeamercolor{block body}{parent=normal text,bg=white}
     \setbeamercolor{author in sidebar}{fg=black}
     \setbeamercolor{title in sidebar}{fg=black}


     \setbeamercolor*{block title}{use=structure,fg=structure.fg,bg=structure.fg!20!bg}
     \setbeamercolor*{block title alerted}{use=alerted text,fg=alerted text.fg,bg=alerted text.fg!20!bg}
     \setbeamercolor*{block title example}{use=example text,fg=example text.fg,bg=example text.fg!20!bg}


     \setbeamercolor{block body}{parent=normal text,use=block title,bg=block title.bg!50!bg}
     \setbeamercolor{block body alerted}{parent=normal text,use=block title alerted,bg=block title alerted.bg!50!bg}
     \setbeamercolor{block body example}{parent=normal text,use=block title example,bg=block title example.bg!50!bg}

     % or ...

     \setbeamercovered{transparent}
     % or whatever (possibly just delete it)
     \logo{\resizebox{!}{1.5cm}{\href{\basename{R}}{\includegraphics{image}}}}
   }

   \mode<handout>
   {
     \usetheme{PaloAlto}
     \usecolortheme{default}
     \setbeamercolor{sidebar}{bg=white, fg=black}
     \setbeamercolor{frametitle}{bg=white, fg=black}
     % or ...
     \setbeamercolor{logo}{bg=white}
     \setbeamercolor{block body}{parent=normal text,bg=white}
     \setbeamercolor{author in sidebar}{fg=black}
     \setbeamercolor{title in sidebar}{fg=black}
     \setbeamercovered{transparent}
     % or whatever (possibly just delete it)
     \logo{}
   }

   \usepackage{epsdice}
   \usepackage[latin1]{inputenc}
   \usepackage{graphics}
   \usepackage{amsmath,eepic,epic}

   \newcommand{\figslide}[3]{
   \begin{frame}
   \frametitle{#1}
     \begin{center}
     \resizebox{!}{2.7in}{\includegraphics{#2}}    
     \end{center}
   {#3}
   \end{frame}
   }

   \newcommand{\fighslide}[4]{
   \begin{frame}
   \frametitle{#1}
     \begin{center}
     \resizebox{!}{#4}{\includegraphics{#2}}    
     \end{center}
   {#3}
   \end{frame}
   }

   \newcommand{\figwref}[1]{
   \href{#1}{\tiny \tt #1}}

   \newcommand{\B}[1]{\beta_{#1}}
   \newcommand{\Bh}[1]{\widehat{\beta}_{#1}}
   \newcommand{\V}{\text{Var}}
   \newcommand{\Cov}{\text{Cov}}
   \newcommand{\Vh}{\widehat{\V}}
   \newcommand{\s}{\sigma}
   \newcommand{\sh}{\widehat{\sigma}}

   \newcommand{\argmax}[1]{\mathop{\text{argmax}}_{#1}}
   \newcommand{\argmin}[1]{\mathop{\text{argmin}}_{#1}}
   \newcommand{\Ee}{\mathbb{E}}
   \newcommand{\Pp}{\mathbb{P}}
   \newcommand{\real}{\mathbb{R}}
   \newcommand{\Ybar}{\overline{Y}}
   \newcommand{\Yh}{\widehat{Y}}
   \newcommand{\Xbar}{\overline{X}}
   \newcommand{\Tr}{\text{Tr}}


   \newcommand{\model}{{\cal M}}

   \newcommand{\figvskip}{-0.7in}
   \newcommand{\fighskip}{-0.3in}
   \newcommand{\figheight}{3.5in}

   \newcommand{\Rcode}[1]{{\bf \tt #1 }}
   \newcommand{\Rtcode}[1]{{\tiny \bf \tt #1 }}
   \newcommand{\Rscode}[1]{{\small \bf \tt #1 }}

   \newcommand{\RR}{{\tt R} \;}
   \newcommand{\basename}[1]{http://stats60.stanford.edu/#1}
   \newcommand{\dataname}[1]{\basename{data/#1}}
   \newcommand{\Rname}[1]{\basename{R/#1}}

   \newcommand{\mycolor}[1]{{\color{blue} #1}}
   \newcommand{\basehref}[2]{\href{\basename{#1}}{\mycolor{#2}}}
   \newcommand{\Rhref}[2]{\href{\basename{R/#1}}{\mycolor{#2}}}
   \newcommand{\datahref}[2]{\href{\dataname{#1}}{\mycolor{#2}}}
   \newcommand{\X}{\pmb{X}}
   \newcommand{\Y}{\pmb{Y}}
   \newcommand{\be}{\pmb{varepsilon}}
   \newcommand{\logit}{\text{logit}}


   \title{Statistics 60: Introduction to Statistical Methods}
   \subtitle{Chapters 13 \& 14: What are the chances?} 
   \author{}% {Jonathan Taylor \\
   %Department of Statistics \\
   %Stanford University
   %}


   \begin{document}

   \begin{frame}
   \titlepage
   \end{frame}

   %%%%%%%%%%%%%%%%%%%%%%%%%%%%%%%%%%%%%%%%%%%%%%%%%%%%%%%%%%%%

   \begin{frame} \frametitle{Probability}

   \begin{block}
   {Frequency definition of chances}
   \begin{itemize}
   \item If you flip a fair coin many times, the long-run
   proportion of heads will be 50\%.

   \item Rolling a fair die, i.e. a cube with faces
   \begin{quote}
   \epsdice{1} \ \epsdice{2} \ \epsdice{3} \ \epsdice{4} \ \epsdice{5} \ \epsdice{6}
   \end{quote}
   will result in a long-run proportion of 1's of $1/6=16 \frac{2}{3}\%$.

   \end{itemize}
   \end{block}
   \end{frame}

   %%%%%%%%%%%%%%%%%%%%%%%%%%%%%%%%%%%%%%%%%%%%%%%%%%%%%%%%%%%%

   \begin{frame} \frametitle{Probability}

   \begin{block}
   {Some rules}
   \begin{description}
     \item[Range of values] Chances are between 0 \% and 100 \%.
     \item[Opposites] The chance of something equals 100 \% minus
     the chance of the opposite thing. Example: the chance
     of not getting a 1 when rolling a die is $83 \frac{1}{3} \%$.
   \end{description}
   \end{block}

   \begin{block}
   {Example: drawing marbles from a box}
   You need to choose a blue marble to win.
   Which box is better?
   \end{block}
   \end{frame}

   %CODE
       % from matplotlib import rc
   % import pylab, numpy as np, sys
   % np.random.seed(0);import random; random.seed(0)
   % sys.path.append('/private/var/folders/dq/4_9bwd013ln6vvf_q110mwrh0000gn/T/tmpBBiJ_x')
   % f=pylab.gcf(); f.set_size_inches(8.0,6.0)
   % datadir ='/private/var/folders/dq/4_9bwd013ln6vvf_q110mwrh0000gn/T/tmpBBiJ_x/data'
   % dx = 0.02
   % X, Y = np.mgrid[0:1:5j,0:1:10j]
   % X += np.random.uniform(0,1,X.shape) * dx - dx / 2
   % Y += np.random.uniform(0,1,Y.shape) * dx - dx / 2
   % 
   % X.shape = -1; Y.shape = -1
   % 
   % g = np.arange(50)
   % np.random.shuffle(g)
   % pylab.scatter(X[g[:30]],Y[g[:30]],s=300, c='b')
   % pylab.scatter(X[g[30:]],Y[g[30:]],s=300, c='r')
   % pylab.gca().set_xticks([])
   % pylab.gca().set_yticks([])
   % pylab.gca().set_xlim([X.min()-0.1,X.max()+0.1])
   % pylab.gca().set_ylim([Y.min()-0.1,Y.max()+0.1])
   % 


   \begin{frame}
   \frametitle{Box \# 1: 30 blue, 20 red}
   \begin{center}
   \resizebox{!}{2.7in}{\includegraphics{./images/inline/3081ff00a3.pdf}}    
   \end{center}

   \end{frame}

   %CODE
       % from matplotlib import rc
   % import pylab, numpy as np, sys
   % np.random.seed(0);import random; random.seed(0)
   % sys.path.append('/private/var/folders/dq/4_9bwd013ln6vvf_q110mwrh0000gn/T/tmpBBiJ_x')
   % f=pylab.gcf(); f.set_size_inches(8.0,6.0)
   % datadir ='/private/var/folders/dq/4_9bwd013ln6vvf_q110mwrh0000gn/T/tmpBBiJ_x/data'
   % dx = 0.02
   % X, Y = np.mgrid[0:1:5j,0:1:1j]
   % X += np.random.uniform(0,1,X.shape) * dx - dx / 2
   % Y += np.random.uniform(0,1,Y.shape) * dx - dx / 2
   % 
   % X.shape = -1; Y.shape = -1
   % 
   % g = np.arange(5)
   % np.random.shuffle(g)
   % pylab.scatter(X[g[:3]],Y[g[:3]],s=300, c='b')
   % pylab.scatter(X[g[3:]],Y[g[3:]],s=300, c='r')
   % pylab.gca().set_xticks([])
   % pylab.gca().set_yticks([])
   % pylab.gca().set_xlim([X.min()-0.1,X.max()+0.1])
   % pylab.gca().set_ylim([Y.min()-0.1,Y.max()+0.1])
   % 


   \begin{frame}
   \frametitle{Box \# 2: 3 blue, 2 red}
   \begin{center}
   \resizebox{!}{2.7in}{\includegraphics{./images/inline/dc8aebfa22.pdf}}    
   \end{center}
   They both have the same chance of winning: 0.6
   \end{frame}

   %%%%%%%%%%%%%%%%%%%%%%%%%%%%%%%%%%%%%%%%%%%%%%%%%%%%%%%%%%%%

   \begin{frame} \frametitle{Probability}

   \begin{block}
   {Drawing marbles}
   \begin{itemize}
     \item When drawing one marble, the important number was
     $$
     \frac{\# \ \text{{\color{blue} blue marbles}}}{\# \ \text{marbles}} = 60\%
     $$
     \item What if you draw 5 marbles without replacement?
     \item With replacement?
   \end{itemize}
   \end{block}
   \end{frame}

   %CODE
       % from matplotlib import rc
   % import pylab, numpy as np, sys
   % np.random.seed(0);import random; random.seed(0)
   % sys.path.append('/private/var/folders/dq/4_9bwd013ln6vvf_q110mwrh0000gn/T/tmpBBiJ_x')
   % f=pylab.gcf(); f.set_size_inches(8.0,6.0)
   % datadir ='/private/var/folders/dq/4_9bwd013ln6vvf_q110mwrh0000gn/T/tmpBBiJ_x/data'
   % dx = 0.02
   % X, Y = np.mgrid[0:1:5j,0:1:1j]
   % X += np.random.uniform(0,1,X.shape) * dx - dx / 2
   % Y += np.random.uniform(0,1,Y.shape) * dx - dx / 2
   % 
   % X.shape = -1; Y.shape = -1
   % 
   % g = np.arange(5)
   % np.random.shuffle(g)
   % pylab.scatter(X[g[:3]],Y[g[:3]],s=300, c='b')
   % pylab.scatter(X[g[4:]],Y[g[4:]],s=300, c='r')
   % pylab.gca().set_xticks([])
   % pylab.gca().set_yticks([])
   % pylab.gca().set_xlim([X.min()-0.1,X.max()+0.1])
   % pylab.gca().set_ylim([Y.min()-0.1,Y.max()+0.1])
   % 


   \begin{frame}
   \frametitle{Box \# 2 after drawing a red}
   \begin{center}
   \resizebox{!}{2.7in}{\includegraphics{./images/inline/a91511f6e4.pdf}}    
   \end{center}
   There are 4 marbles in box, ready to draw without replacement
   \end{frame}

   %CODE
       % from matplotlib import rc
   % import pylab, numpy as np, sys
   % np.random.seed(0);import random; random.seed(0)
   % sys.path.append('/private/var/folders/dq/4_9bwd013ln6vvf_q110mwrh0000gn/T/tmpBBiJ_x')
   % f=pylab.gcf(); f.set_size_inches(8.0,6.0)
   % datadir ='/private/var/folders/dq/4_9bwd013ln6vvf_q110mwrh0000gn/T/tmpBBiJ_x/data'
   % dx = 0.02
   % X, Y = np.mgrid[0:1:5j,0:1:1j]
   % X += np.random.uniform(0,1,X.shape) * dx - dx / 2
   % Y += np.random.uniform(0,1,Y.shape) * dx - dx / 2
   % 
   % X.shape = -1; Y.shape = -1
   % 
   % g = np.arange(5)
   % np.random.shuffle(g)
   % pylab.scatter(X[g[:3]],Y[g[:3]],s=300, c='b')
   % # pylab.scatter(X[g[4:]],Y[g[4:]],s=300, c='b')
   % pylab.gca().set_xticks([])
   % pylab.gca().set_yticks([])
   % pylab.gca().set_xlim([X.min()-0.1,X.max()+0.1])
   % pylab.gca().set_ylim([Y.min()-0.1,Y.max()+0.1])
   % 


   \begin{frame}
   \frametitle{Box \# 2 after drawing two reds}
   \begin{center}
   \resizebox{!}{2.7in}{\includegraphics{./images/inline/767bcfad5c.pdf}}    
   \end{center}
   Next draw will be a blue!
   \end{frame}

   %CODE
       % from matplotlib import rc
   % import pylab, numpy as np, sys
   % np.random.seed(0);import random; random.seed(0)
   % sys.path.append('/private/var/folders/dq/4_9bwd013ln6vvf_q110mwrh0000gn/T/tmpBBiJ_x')
   % f=pylab.gcf(); f.set_size_inches(8.0,6.0)
   % datadir ='/private/var/folders/dq/4_9bwd013ln6vvf_q110mwrh0000gn/T/tmpBBiJ_x/data'
   % dx = 0.02
   % X, Y = np.mgrid[0:1:5j,0:1:1j]
   % X += np.random.uniform(0,1,X.shape) * dx - dx / 2
   % Y += np.random.uniform(0,1,Y.shape) * dx - dx / 2
   % 
   % X.shape = -1; Y.shape = -1
   % 
   % g = np.arange(5)
   % np.random.shuffle(g)
   % pylab.scatter(X[g[:3]],Y[g[:3]],s=300, c='b')
   % pylab.scatter(X[g[3:]],Y[g[3:]],s=300, c='r')
   % pylab.gca().set_xticks([])
   % pylab.gca().set_yticks([])
   % pylab.gca().set_xlim([X.min()-0.1,X.max()+0.1])
   % pylab.gca().set_ylim([Y.min()-0.1,Y.max()+0.1])
   % 


   \begin{frame}
   \frametitle{Box \# 2 with red replaced}
   \begin{center}
   \resizebox{!}{2.7in}{\includegraphics{./images/inline/dc8aebfa22.pdf}}    
   \end{center}
   There are again 5 marbles, ready to draw with replacement
   \end{frame}

   %CODE
       % from matplotlib import rc
   % import pylab, numpy as np, sys
   % np.random.seed(0);import random; random.seed(0)
   % sys.path.append('/private/var/folders/dq/4_9bwd013ln6vvf_q110mwrh0000gn/T/tmpBBiJ_x')
   % f=pylab.gcf(); f.set_size_inches(8.0,6.0)
   % datadir ='/private/var/folders/dq/4_9bwd013ln6vvf_q110mwrh0000gn/T/tmpBBiJ_x/data'
   % dx = 0.02
   % X, Y = np.mgrid[0:1:5j,0:1:1j]
   % X += np.random.uniform(0,1,X.shape) * dx - dx / 2
   % Y += np.random.uniform(0,1,Y.shape) * dx - dx / 2
   % 
   % X.shape = -1; Y.shape = -1
   % 
   % g = np.arange(5)
   % np.random.shuffle(g)
   % pylab.scatter(X[g[:3]],Y[g[:3]],s=300, c='b')
   % pylab.scatter(X[g[3:]],Y[g[3:]],s=300, c='r')
   % pylab.gca().set_xticks([])
   % pylab.gca().set_yticks([])
   % pylab.gca().set_xlim([X.min()-0.1,X.max()+0.1])
   % pylab.gca().set_ylim([Y.min()-0.1,Y.max()+0.1])
   % 


   \begin{frame}
   \frametitle{Box \# 2}
   \begin{center}
   \resizebox{!}{2.7in}{\includegraphics{./images/inline/dc8aebfa22.pdf}}    
   \end{center}
   The chance is always 60\% \dots
   \end{frame}

   %%%%%%%%%%%%%%%%%%%%%%%%%%%%%%%%%%%%%%%%%%%%%%%%%%%%%%%%%%%%

   \begin{frame} \frametitle{Probability}

   \begin{block}
   {Drawing five marbles from Box \# 1 or \# 2}
   \begin{itemize}
     \item If drawing without replacement, Box \# 2 will be easier to win.
     \item We're even guaranteed to win with Box \# 2.
   \end{itemize}
   \end{block}
   \end{frame}

   %CODE
       % from matplotlib import rc
   % import pylab, numpy as np, sys
   % np.random.seed(0);import random; random.seed(0)
   % sys.path.append('/private/var/folders/dq/4_9bwd013ln6vvf_q110mwrh0000gn/T/tmpBBiJ_x')
   % f=pylab.gcf(); f.set_size_inches(8.0,6.0)
   % datadir ='/private/var/folders/dq/4_9bwd013ln6vvf_q110mwrh0000gn/T/tmpBBiJ_x/data'
   % import numpy as np, pylab
   % 
   % def ticket(t,x,y,s):
   % 
   %     xf, yf = pylab.poly_between([x-s/2.,x+s/2.],[y-s/2.,y-s/2.],[y+s/2.,y+s/2.])
   %     tt = pylab.text(x,y,t)
   %     return xf, yf, tt
   % 
   % for i, c in zip(range(3), ['red', 'blue', 'yellow']):
   %     xf, yf, tt = ticket('',i,0,0.8)
   %     pylab.fill(xf,yf, facecolor=c, edgecolor='black', alpha=0.4, linewidth=2)
   %     tt.set_size(50)
   % 
   % pylab.gca().set_xticks([])
   % pylab.gca().set_yticks([])
   % pylab.gca().set_ylim([-0.5,0.5])
   % 


   \begin{frame}
   \frametitle{Drawing tickets}
   \begin{center}
   \resizebox{!}{2.7in}{\includegraphics{./images/inline/7fc93047e7.pdf}}    
   \end{center}
   Suppose we draw the {\color{blue} blue ticket}
   \end{frame}

   %CODE
       % from matplotlib import rc
   % import pylab, numpy as np, sys
   % np.random.seed(0);import random; random.seed(0)
   % sys.path.append('/private/var/folders/dq/4_9bwd013ln6vvf_q110mwrh0000gn/T/tmpBBiJ_x')
   % f=pylab.gcf(); f.set_size_inches(8.0,6.0)
   % datadir ='/private/var/folders/dq/4_9bwd013ln6vvf_q110mwrh0000gn/T/tmpBBiJ_x/data'
   % import numpy as np, pylab
   % 
   % def ticket(t,x,y,s):
   % 
   %     xf, yf = pylab.poly_between([x-s/2.,x+s/2.],[y-s/2.,y-s/2.],[y+s/2.,y+s/2.])
   %     tt = pylab.text(x,y,t)
   %     return xf, yf, tt
   % 
   % for i, c in zip(range(3), ['red', 'blue', 'yellow']):
   %     if c != 'blue':
   %        xf, yf, tt = ticket('',i,0,0.8)
   %        pylab.fill(xf,yf, facecolor=c, edgecolor='black', alpha=0.4, linewidth=2)
   %        tt.set_size(50)
   % 
   % pylab.gca().set_xticks([])
   % pylab.gca().set_yticks([])
   % pylab.gca().set_ylim([-0.5,0.5])
   % 


   \begin{frame}
   \frametitle{Drawing tickets}
   \begin{center}
   \resizebox{!}{2.7in}{\includegraphics{./images/inline/731addcd92.pdf}}    
   \end{center}
   This is the box if we are drawing {\bf without replacement}.
   \end{frame}

   %CODE
       % from matplotlib import rc
   % import pylab, numpy as np, sys
   % np.random.seed(0);import random; random.seed(0)
   % sys.path.append('/private/var/folders/dq/4_9bwd013ln6vvf_q110mwrh0000gn/T/tmpBBiJ_x')
   % f=pylab.gcf(); f.set_size_inches(8.0,6.0)
   % datadir ='/private/var/folders/dq/4_9bwd013ln6vvf_q110mwrh0000gn/T/tmpBBiJ_x/data'
   % import numpy as np, pylab
   % 
   % def ticket(t,x,y,s):
   % 
   %     xf, yf = pylab.poly_between([x-s/2.,x+s/2.],[y-s/2.,y-s/2.],[y+s/2.,y+s/2.])
   %     tt = pylab.text(x,y,t)
   %     return xf, yf, tt
   % 
   % for i, c in zip(range(3), ['red', 'blue', 'yellow']):
   %     xf, yf, tt = ticket('',i,0,0.8)
   %     pylab.fill(xf,yf, facecolor=c, edgecolor='black', alpha=0.4, linewidth=2)
   %     tt.set_size(50)
   % 
   % pylab.gca().set_xticks([])
   % pylab.gca().set_yticks([])
   % pylab.gca().set_ylim([-0.5,0.5])
   % 


   \begin{frame}
   \frametitle{Drawing tickets}
   \begin{center}
   \resizebox{!}{2.7in}{\includegraphics{./images/inline/7fc93047e7.pdf}}    
   \end{center}
   This is the box if we are drawing {\bf with replacement}.
   \end{frame}

   %%%%%%%%%%%%%%%%%%%%%%%%%%%%%%%%%%%%%%%%%%%%%%%%%%%%%%%%%%%%

   \begin{frame} \frametitle{Probability}

   \begin{block}
   {Conditional probability}
   \begin{itemize}
     \item Observing some information can {\em change} the chances
     of something.
     \item Example: if drawing without replacement, suppose
     the first draw was red. What are the chances
     a blue marble is drawn on the second draw?
     \item What if we draw with replacement?
     \item In this examples, we are {\em given} that the
     first draw was red. These chances are {\em conditioned}
     on knowing the first draw was red.
   \end{itemize}
   \end{block}
   \end{frame}

   %%%%%%%%%%%%%%%%%%%%%%%%%%%%%%%%%%%%%%%%%%%%%%%%%%%%%%%%%%%%

   \begin{frame} \frametitle{Probability}

   \begin{block}
   {Multiplication rule}
   The chance that two things will both happen equals
   the chance that the first will happen, multiplied
   by the chance that the second will happen given the
   first has happened.
   \end{block}
   \end{frame}

   %%%%%%%%%%%%%%%%%%%%%%%%%%%%%%%%%%%%%%%%%%%%%%%%%%%%%%%%%%%%

   \begin{frame} \frametitle{Probability}

   \begin{block}
   {Example}
   \begin{description}
   \item[Q] What is the probability the first blue marble drawn
   is on the second draw?
   \item[A] If the first blue marble drawn was the second, then we know
     \begin{itemize}
     \item the first was red;

     \item the second was blue.
     \end{itemize}
     By the multiplication rule
     $$
     \text{chances} = \frac{2}{5} \times \frac{3}{4} = \frac{3}{10}
     $$
   \end{description}
   \end{block}
   \end{frame}

   %%%%%%%%%%%%%%%%%%%%%%%%%%%%%%%%%%%%%%%%%%%%%%%%%%%%%%%%%%%%

   \begin{frame} \frametitle{Probability}

   \begin{block}
   {Mathematical notation}
   \begin{itemize}
   \item ``first blue drawn is on the second draw'' is called an {\em event};
   \item ``first draw is red'' and ``second draw is blue'' are also events;
     \item We usually write $P$ for ``chances''. For an event $E$
   $$
   P(E) = \text{chances $E$ occurs}.
   $$
   \item Conditional probability of an event $A$ given $B$ is written as
   $$
   P(A|B) = \text{chances $A$ occurs given $B$ occurs}.
   $$
   \item Multiplication rule
   $$
   P(\text{$A$ and $B$ occur}) = P(A|B) \times P(B).
   $$
   \end{itemize}
   \end{block}
   \end{frame}

   %%%%%%%%%%%%%%%%%%%%%%%%%%%%%%%%%%%%%%%%%%%%%%%%%%%%%%%%%%%%

   \begin{frame} \frametitle{Probability}

   \begin{block}
   {Law of total mass}
   \begin{itemize}
   \item    The chances of something occuring are 100\%.
   \item Example: when we draw marbles, the chances we draw a marble
   whose color is blue or red is 100 \%.
   \item In mathematical notation, we often use $S$ for ``something''
   or the ``sample space''
   $$
   P(S) = 100\% \qquad (= 1)
   $$
   \end{itemize}
   \end{block}
   \end{frame}

   %%%%%%%%%%%%%%%%%%%%%%%%%%%%%%%%%%%%%%%%%%%%%%%%%%%%%%%%%%%%

   \begin{frame} \frametitle{Probability}

   \begin{block}
   {Example: law of total mass}
   \begin{itemize}
   \item  When drawing from Box \# 2 without replacement,
   we will draw a blue ball within the first three draws.
   $$
   P(\text{one of the first three balls is blue}) = 100 \%
   $$
   \item We can verify the law of total mass
   $$
   \begin{aligned}
   P(\text{first blue ball is on draw \# 1}) &= \frac{3}{5} \\
   P(\text{first blue ball is on draw \# 2}) &= \frac{2}{5} \times \frac{3}{4} = \frac{3}{10}  \\
   P(\text{first blue ball is on draw \# 3}) &= \frac{2}{5} \times \frac{1}{4} = \frac{1}{10}  \\
   \end{aligned}
   $$
   \item Summing the probablities $\frac{3}{5} + \frac{3}{10} +
   \frac{1}{10} = 1.$
   \end{itemize}

   \end{block}
   \end{frame}

   %%%%%%%%%%%%%%%%%%%%%%%%%%%%%%%%%%%%%%%%%%%%%%%%%%%%%%%%%%%%

   \begin{frame} \frametitle{Probability}

   \begin{block}
   {Addition rule}
   \begin{description}
   \item[Q] When can we add probabilities of different events?
   \item[A] When the events are {\em disjoint} or {\em mutually exclusive}.
   \item[Example] When rolling a die, the events
   $E_1=$roll is \epsdice{3}, $E_2=$roll is \epsdice{4} \, are
   mutually exclusive
   because the result of the roll cannot be \epsdice{3} \, and \epsdice{4} \,
   simultaneously.
   \end{description}
   \end{block}
   \end{frame}

   %CODE
       % from matplotlib import rc
   % import pylab, numpy as np, sys
   % np.random.seed(0);import random; random.seed(0)
   % sys.path.append('/private/var/folders/dq/4_9bwd013ln6vvf_q110mwrh0000gn/T/tmpBBiJ_x')
   % f=pylab.gcf(); f.set_size_inches(6.0,6.0)
   % datadir ='/private/var/folders/dq/4_9bwd013ln6vvf_q110mwrh0000gn/T/tmpBBiJ_x/data'
   % import numpy as np, pylab, matplotlib
   % 
   % import matplotlib
   % cir = matplotlib.patches.Circle
   % 
   % a = pylab.gca()
   % # add a circle
   % E1 = cir((0.5,0.5), 0.4,ec="black", facecolor='yellow',lw=2, alpha=0.4)
   % E2 = cir((-0.2,-0.2), 0.4,ec="black", facecolor='blue',lw=2, alpha=0.4)
   % a.add_patch(E1)
   % a.add_patch(E2)
   % a.set_xticks([])
   % a.set_yticks([])
   % a.set_xlim([-0.7,1])
   % a.set_ylim([-0.7,1])
   % 
   % # pylab.draw()
   % 


   \begin{frame}
   \frametitle{Mutually exclusive events}
   \begin{center}
   \resizebox{!}{2.7in}{\includegraphics{./images/inline/c50b5aedbe.pdf}}    
   \end{center}

   \end{frame}

   %CODE
       % from matplotlib import rc
   % import pylab, numpy as np, sys
   % np.random.seed(0);import random; random.seed(0)
   % sys.path.append('/private/var/folders/dq/4_9bwd013ln6vvf_q110mwrh0000gn/T/tmpBBiJ_x')
   % f=pylab.gcf(); f.set_size_inches(6.0,6.0)
   % datadir ='/private/var/folders/dq/4_9bwd013ln6vvf_q110mwrh0000gn/T/tmpBBiJ_x/data'
   % import numpy as np, pylab, matplotlib
   % 
   % import matplotlib
   % 
   % rc('text', usetex=True)
   % 
   % cir = matplotlib.patches.Circle
   % 
   % a = pylab.gca()
   % # add a circle
   % E1 = cir((0.5,0.5), 0.4,ec="black", facecolor='yellow',lw=2, alpha=0.4)
   % E2 = cir((0.2,0.2), 0.4,ec="black", facecolor='blue',lw=2, alpha=0.4)
   % a.add_patch(E1)
   % a.add_patch(E2)
   % a.set_xticks([])
   % a.set_yticks([])
   % a.set_xlim([-0.3,1])
   % a.set_ylim([-0.3,1])
   % 


   \begin{frame}
   \frametitle{Non-mutually exclusive events}
   \begin{center}
   \resizebox{!}{2.7in}{\includegraphics{./images/inline/c1b6d1a77a.pdf}}    
   \end{center}

   \end{frame}

   %%%%%%%%%%%%%%%%%%%%%%%%%%%%%%%%%%%%%%%%%%%%%%%%%%%%%%%%%%%%

   \begin{frame} \frametitle{Probability}

   \begin{block}
   {Addition rule}
   \begin{itemize}
   \item    If the events $E_1, E_2$ are mutually exclusive, then
   $$
   P(\text{$E_1$ or $E_2$}) = P(E_1) + P(E_2).
   $$
   \item This rule works for more than two: if $[E_1, \dots, E_n]$
     are mutually exclusive, then
   $$
   P(\text{$E_1$ or $E_2$ or \dots or $E_n$}) = \sum_{i=1}^n P(E_i).
   $$

   \end{itemize}
   \end{block}
   \end{frame}

   %%%%%%%%%%%%%%%%%%%%%%%%%%%%%%%%%%%%%%%%%%%%%%%%%%%%%%%%%%%%

   \begin{frame} \frametitle{Probability}

   \begin{block}
   {Addition rule}
   \begin{itemize}
   \item The events $E_1, E_2$ are mutually exclusive if $E_1 \cap E_2$ is empty.
   \item We often write $\text{$E_1$ or $E_2$}$ as $E_1 \cup E_2$.
   $$
   P(\text{$E_1$ or $E_2$}) = P(E_1 \cup E_2)
   $$

   \item From the Venn diagram, we can deduce the general rule
   $$
   P(E_1 \cup E_2) = P(E_1) + P(E_2) - P(E_1 \cap E_2)
   $$

   \item There is a rule for more than two events \dots
   \end{itemize}
   \end{block}
   \end{frame}

   %CODE
       % from matplotlib import rc
   % import pylab, numpy as np, sys
   % np.random.seed(0);import random; random.seed(0)
   % sys.path.append('/private/var/folders/dq/4_9bwd013ln6vvf_q110mwrh0000gn/T/tmpBBiJ_x')
   % f=pylab.gcf(); f.set_size_inches(6.0,6.0)
   % datadir ='/private/var/folders/dq/4_9bwd013ln6vvf_q110mwrh0000gn/T/tmpBBiJ_x/data'
   % import matplotlib
   % cir = matplotlib.patches.Circle
   % 
   % a = pylab.gca()
   % # add a circle
   % E1 = cir((0.5,0.5), 0.4,ec="black", facecolor='yellow',lw=2, alpha=0.4)
   % E2 = cir((0.2,0.2), 0.4,ec="black", facecolor='blue',lw=2, alpha=0.4)
   % E3 = cir((0.2,0.5), 0.4,ec="black", facecolor='red',lw=2, alpha=0.4)
   % a.add_patch(E1)
   % a.add_patch(E2)
   % a.add_patch(E3)
   % a.set_xticks([])
   % a.set_yticks([])
   % a.set_xlim([-0.3,1])
   % a.set_ylim([-0.3,1])
   % 


   \begin{frame}
   \frametitle{Three events}
   \begin{center}
   \resizebox{!}{2.7in}{\includegraphics{./images/inline/7f9c9c124d.pdf}}    
   \end{center}
   There are 3 2-way intersections and one 3-way intersection...
   \end{frame}

   %%%%%%%%%%%%%%%%%%%%%%%%%%%%%%%%%%%%%%%%%%%%%%%%%%%%%%%%%%%%

   \begin{frame} \frametitle{Probability}

   \begin{block}
   {Addition rule}
   \begin{itemize}
   \item Because chances (or probabilities) are between 0\% and 100\%,
   we can deduce
   $$
   P(E_1 \cup E_2) \leq P(E_1) + P(E_2).
   $$
   \item This rule holds for any number of events
   $$
   \begin{aligned}
   P(\text{$E_1$ or $E_2$ or \dots or $E_n$}) &= P \left(\bigcup_{i=1}^n E_i \right) \\
   & \leq \sum_{i=1}^n P(E_i).
   \end{aligned}
   $$

   \end{itemize}
   \end{block}
   \end{frame}

   %%%%%%%%%%%%%%%%%%%%%%%%%%%%%%%%%%%%%%%%%%%%%%%%%%%%%%%%%%%%

   \begin{frame} \frametitle{Probability}

   \begin{block}
   {Multiplication rule \& independence}
   \begin{itemize}
   \item Intuitively, an event $A$ is independent of $B$ if given $A$,
   the odds of $B$ are unaffected.
   \item In mathematical notation, we express this
   notion as
   $$
   P(B|A)=P(B)
   $$
   \item If this is true, we say $A$ and $B$ are {\em independent}.
     \item Otherwise, $A$ and $B$ are {\em dependent}.
   \item The multiplication rule, combined with independence tells us
   $$
   P(A \cap B) = P(B|A) \times P(A) = P(B) \times P(A).
   $$
   \end{itemize}
   \end{block}
   \end{frame}

   %%%%%%%%%%%%%%%%%%%%%%%%%%%%%%%%%%%%%%%%%%%%%%%%%%%%%%%%%%%%

   \begin{frame} \frametitle{Probability}

   \begin{block}
   {Drawing marbles}
   \begin{itemize}
   \item When drawing marbles {\em with replacement}
   the events
   $$
   \begin{aligned}
   A &= \text{first draw is red} \\
   B &= \text{second draw is blue}
   \end{aligned}
   $$
   are {\em independent}
   \item We can even conclude that the draws are independent in this case.
   \item When drawing {\em without replacement}
   the events $A$ and $B$ are dependent.
   \end{itemize}
   \end{block}
   \end{frame}

   %%%%%%%%%%%%%%%%%%%%%%%%%%%%%%%%%%%%%%%%%%%%%%%%%%%%%%%%%%%%

   \begin{frame} \frametitle{Probability}

   \begin{block}
   {Counting}
   \begin{itemize}
   \item When performing an experiment where each outcome
   is equally likely, we can compute probabilities by counting.
   \item Example: when rolling two dice, what is the
   probability of obtaining a sum of 9?
   \item We call these counting problems {\em combinatorial}.
   \item For such experiments
   $$
   P(E) = \frac{\# E}{\# S}
   $$
   where $S$ is the set of all possible outcomes (called the {\em sample space}).
   \end{itemize}
   \end{block}
   \end{frame}

   %%%%%%%%%%%%%%%%%%%%%%%%%%%%%%%%%%%%%%%%%%%%%%%%%%%%%%%%%%%%

   \begin{frame} \frametitle{Probability}

   \begin{block}
   {Rolling two dice: all possible outcomes}
   \begin{table}
     \centering
   \begin{tabular}{cccccc}
   \epsdice{1} \, \epsdice[black]{1} & \epsdice{1} \, \epsdice[black]{2} & \epsdice{1} \, \epsdice[black]{3} & \epsdice{1} \, \epsdice[black]{4} & \epsdice{1} \, \epsdice[black]{5} & \epsdice{1} \, \epsdice[black]{6} \\
   \epsdice{2} \, \epsdice[black]{1} & \epsdice{2} \, \epsdice[black]{2} & \epsdice{2} \, \epsdice[black]{3} & \epsdice{2} \, \epsdice[black]{4} & \epsdice{2} \, \epsdice[black]{5} & \epsdice{2} \, \epsdice[black]{6} \\
   \epsdice{3} \, \epsdice[black]{1} & \epsdice{3} \, \epsdice[black]{2} & \epsdice{3} \, \epsdice[black]{3} & \epsdice{3} \, \epsdice[black]{4} & \epsdice{3} \, \epsdice[black]{5} & \epsdice{3} \, \epsdice[black]{6} \\
   \epsdice{4} \, \epsdice[black]{1} & \epsdice{4} \, \epsdice[black]{2} & \epsdice{4} \, \epsdice[black]{3} & \epsdice{4} \, \epsdice[black]{4} & \epsdice{4} \, \epsdice[black]{5} & \epsdice{4} \, \epsdice[black]{6} \\
   \epsdice{5} \, \epsdice[black]{1} & \epsdice{5} \, \epsdice[black]{2} & \epsdice{5} \, \epsdice[black]{3} & \epsdice{5} \, \epsdice[black]{4} & \epsdice{5} \, \epsdice[black]{5} & \epsdice{5} \, \epsdice[black]{6} \\
   \epsdice{6} \, \epsdice[black]{1} & \epsdice{6} \, \epsdice[black]{2} & \epsdice{6} \, \epsdice[black]{3} & \epsdice{6} \, \epsdice[black]{4} & \epsdice{6} \, \epsdice[black]{5} & \epsdice{6} \, \epsdice[black]{6} \\
   \end{tabular}
   \end{table}

   \end{block}
   \end{frame}

   %%%%%%%%%%%%%%%%%%%%%%%%%%%%%%%%%%%%%%%%%%%%%%%%%%%%%%%%%%%%

   \begin{frame} \frametitle{Probability}

   \begin{block}
   {What are the chances the sum will be equal to 9?}
   \begin{table}
     \centering
   \begin{tabular}{cccccc}
   \epsdice{1} \, \epsdice[black]{1} & \epsdice{1} \, \epsdice[black]{2} & \epsdice{1} \, \epsdice[black]{3} & \epsdice{1} \, \epsdice[black]{4} & \epsdice{1} \, \epsdice[black]{5} & \epsdice{1} \, \epsdice[black]{6} \\
   \epsdice{2} \, \epsdice[black]{1} & \epsdice{2} \, \epsdice[black]{2} & \epsdice{2} \, \epsdice[black]{3} & \epsdice{2} \, \epsdice[black]{4} & \epsdice{2} \, \epsdice[black]{5} & \epsdice{2} \, \epsdice[black]{6} \\
   \epsdice{3} \, \epsdice[black]{1} & \epsdice{3} \, \epsdice[black]{2} & \epsdice{3} \, \epsdice[black]{3} & \epsdice{3} \, \epsdice[black]{4} & \epsdice{3} \, \epsdice[black]{5} & {\color{red} \fbox{\epsdice{3} \, \epsdice[black]{6}}} \\
   \epsdice{4} \, \epsdice[black]{1} & \epsdice{4} \, \epsdice[black]{2} & \epsdice{4} \, \epsdice[black]{3} & \epsdice{4} \, \epsdice[black]{4} & {\color{red} \fbox{\epsdice{4} \, \epsdice[black]{5}}} & \epsdice{4} \, \epsdice[black]{6} \\
   \epsdice{5} \, \epsdice[black]{1} & \epsdice{5} \, \epsdice[black]{2} & \epsdice{5} \, \epsdice[black]{3} & {\color{red} \fbox{\epsdice{5} \, \epsdice[black]{4}}} & \epsdice{5} \, \epsdice[black]{5} & \epsdice{5} \, \epsdice[black]{6} \\
   \epsdice{6} \, \epsdice[black]{1} & \epsdice{6} \, \epsdice[black]{2} & {\color{red} \fbox{\epsdice{6} \, \epsdice[black]{3}}} & \epsdice{6} \, \epsdice[black]{4} & \epsdice{6} \, \epsdice[black]{5} & \epsdice{6} \, \epsdice[black]{6} \\
   \end{tabular}
   \end{table}
   There are 4 outcomes whose sum is 9. Therefore, the chances are $\frac{4}{36}=\frac{1}{9}$.
   \end{block}
   \end{frame}

   %%%%%%%%%%%%%%%%%%%%%%%%%%%%%%%%%%%%%%%%%%%%%%%%%%%%%%%%%%%%

   \begin{frame} \frametitle{Probability}

   \begin{block}
   {What are the chances the sum will be greater than or equal to 7?}
   \begin{table}
     \centering
   \begin{tabular}{cccccc}
   \epsdice{1} \, \epsdice[black]{1} & \epsdice{1} \, \epsdice[black]{2} & \epsdice{1} \, \epsdice[black]{3} & \epsdice{1} \, \epsdice[black]{4} & \epsdice{1} \, \epsdice[black]{5} & {\color{red} \fbox{\epsdice{1} \, \epsdice[black]{6}}} \\
   \epsdice{2} \, \epsdice[black]{1} & \epsdice{2} \, \epsdice[black]{2} & \epsdice{2} \, \epsdice[black]{3} & \epsdice{2} \, \epsdice[black]{4} & {\color{red} \fbox{\epsdice{2} \, \epsdice[black]{5}}} & {\color{red} \fbox{\epsdice{2} \, \epsdice[black]{6}}} \\
   \epsdice{3} \, \epsdice[black]{1} & \epsdice{3} \, \epsdice[black]{2} & \epsdice{3} \, \epsdice[black]{3} & {\color{red} \fbox{\epsdice{3} \, \epsdice[black]{4}}} & {\color{red} \fbox{\epsdice{3} \, \epsdice[black]{5}}} & {\color{red} \fbox{\epsdice{3} \, \epsdice[black]{6}}} \\
   \epsdice{4} \, \epsdice[black]{1} & \epsdice{4} \, \epsdice[black]{2} & {\color{red} \fbox{\epsdice{4} \, \epsdice[black]{3}}} & {\color{red} \fbox{\epsdice{4} \, \epsdice[black]{4}}} & {\color{red} \fbox{\epsdice{4} \, \epsdice[black]{5}}} & {\color{red} \fbox{\epsdice{4} \, \epsdice[black]{6}}} \\
   \epsdice{5} \, \epsdice[black]{1} & {\color{red} \fbox{\epsdice{5} \, \epsdice[black]{2}}} & {\color{red} \fbox{\epsdice{5} \, \epsdice[black]{3}}} & {\color{red} \fbox{\epsdice{5} \, \epsdice[black]{4}}} & {\color{red} \fbox{\epsdice{5} \, \epsdice[black]{5}}} & {\color{red} \fbox{\epsdice{5} \, \epsdice[black]{6}}} \\
   {\color{red} \fbox{\epsdice{6} \, \epsdice[black]{1}}} & {\color{red} \fbox{\epsdice{6} \, \epsdice[black]{2}}} & {\color{red} \fbox{\epsdice{6} \, \epsdice[black]{3}}} & {\color{red} \fbox{\epsdice{6} \, \epsdice[black]{4}}} & {\color{red} \fbox{\epsdice{6} \, \epsdice[black]{5}}} & {\color{red} \fbox{\epsdice{6} \, \epsdice[black]{6}}} \\
   \end{tabular}
   \end{table}
   There are 21 outcomes whose sum is greater than or equal to 7. Therefore, the chances are $\frac{21}{36}=\frac{7}{12}$.
   \end{block}
   \end{frame}

   %%%%%%%%%%%%%%%%%%%%%%%%%%%%%%%%%%%%%%%%%%%%%%%%%%%%%%%%%%%%

   \begin{frame} \frametitle{Probability}

   \begin{block}
   {What are the chances the sum will be less than 7?}
   \begin{table}
     \centering
   \begin{tabular}{cccccc}
   {\color{blue} \fbox{\epsdice{1} \, \epsdice[black]{1}}} & {\color{blue} \fbox{\epsdice{1} \, \epsdice[black]{2}}} & {\color{blue} \fbox{\epsdice{1} \, \epsdice[black]{3}}} & {\color{blue} \fbox{\epsdice{1} \, \epsdice[black]{4}}} & {\color{blue} \fbox{\epsdice{1} \, \epsdice[black]{5}}} & \epsdice{1} \, \epsdice[black]{6} \\
   {\color{blue} \fbox{\epsdice{2} \, \epsdice[black]{1}}} & {\color{blue} \fbox{\epsdice{2} \, \epsdice[black]{2}}} & {\color{blue} \fbox{\epsdice{2} \, \epsdice[black]{3}}} & {\color{blue} \fbox{\epsdice{2} \, \epsdice[black]{4}}} & \epsdice{2} \, \epsdice[black]{5} & \epsdice{2} \, \epsdice[black]{6} \\
   {\color{blue} \fbox{\epsdice{3} \, \epsdice[black]{1}}} & {\color{blue} \fbox{\epsdice{3} \, \epsdice[black]{2}}} & {\color{blue} \fbox{\epsdice{3} \, \epsdice[black]{3}}} & \epsdice{3} \, \epsdice[black]{4} & \epsdice{3} \, \epsdice[black]{5} & \epsdice{3} \, \epsdice[black]{6} \\
   {\color{blue} \fbox{\epsdice{4} \, \epsdice[black]{1}}} & {\color{blue} \fbox{\epsdice{4} \, \epsdice[black]{2}}} & \epsdice{4} \, \epsdice[black]{3} & \epsdice{4} \, \epsdice[black]{4} & \epsdice{4} \, \epsdice[black]{5} & \epsdice{4} \, \epsdice[black]{6} \\
   {\color{blue} \fbox{\epsdice{5} \, \epsdice[black]{1}}} & \epsdice{5} \, \epsdice[black]{2} & \epsdice{5} \, \epsdice[black]{3} & \epsdice{5} \, \epsdice[black]{4} & \epsdice{5} \, \epsdice[black]{5} & \epsdice{5} \, \epsdice[black]{6} \\
   \epsdice{6} \, \epsdice[black]{1} & \epsdice{6} \, \epsdice[black]{2} & \epsdice{6} \, \epsdice[black]{3} & \epsdice{6} \, \epsdice[black]{4} & \epsdice{6} \, \epsdice[black]{5} & \epsdice{6} \, \epsdice[black]{6} \\
   \end{tabular}
   \end{table}
   There chances that the sum is greater than or equal to 7 are $\frac{7}{12}$.
   Therefore, by the
   ``opposite'' rule, the chances are chances are $1-\frac{7}{12}=\frac{5}{12}.$
   \end{block}
   \end{frame}

   %%%%%%%%%%%%%%%%%%%%%%%%%%%%%%%%%%%%%%%%%%%%%%%%%%%%%%%%%%%%

   \begin{frame} \frametitle{Probability}

   \begin{block}
   {Complement of an event}
   \begin{itemize}
   \item Formally, the ``opposite'' rule is the rule
   of {\em complements}.
   \item We write the complement of an event $E$ as $E^c$
   $$
   P(\text{not} \, E) = P(E^c).
   $$
   \item The rule of {\em complements} says
   $$
   P(E^c) = 1 - P(E)
   $$
   \end{itemize}
   \end{block}
   \end{frame}

   %CODE
       % from matplotlib import rc
   % import pylab, numpy as np, sys
   % np.random.seed(0);import random; random.seed(0)
   % sys.path.append('/private/var/folders/dq/4_9bwd013ln6vvf_q110mwrh0000gn/T/tmpBBiJ_x')
   % f=pylab.gcf(); f.set_size_inches(6.0,6.0)
   % datadir ='/private/var/folders/dq/4_9bwd013ln6vvf_q110mwrh0000gn/T/tmpBBiJ_x/data'
   % import matplotlib
   % 
   % rc('text', usetex=True)
   % cir = matplotlib.patches.Circle
   % 
   % 
   % a = pylab.gca()
   % a.set_alpha(0.4)
   % E1 = cir((0.5,0.5), 0.4,ec="black", facecolor='yellow', lw=2, alpha=0.4)
   % a.add_patch(E1)
   % 
   % a.set_xticks([])
   % a.set_yticks([])
   % a.set_xlim([-0.3,1])
   % a.set_ylim([-0.3,1])
   % 
   % t = pylab.text(0.5, 0.5, '$E$', size=40, color='black', va='center', ha='center')
   % t = pylab.text(-0.05, -0.05, r'$E^c$', size=40, color='black', va='center', ha='center')
   % f = pylab.gcf()
   % f.text(0.8, 0.04, '$S$', size=40, ha='center', va='center')
   % 


   \begin{frame}
   \frametitle{An event $E$ and its complement $E^c$}
   \begin{center}
   \resizebox{!}{2.7in}{\includegraphics{./images/inline/524591d70e.pdf}}    
   \end{center}

   \end{frame}

   %%%%%%%%%%%%%%%%%%%%%%%%%%%%%%%%%%%%%%%%%%%%%%%%%%%%%%%%%%%%

   \begin{frame} \frametitle{Probability}

   \begin{block}
   {Properties of complements}
   \begin{itemize}
   \item For any event $E$, $E$ and $E^c$ are mutually exclusive.
   \item For any event $E$, $S = E \cup E^c$.
   \item For any two events $A, B$
   $$
   \begin{aligned}
   B &= B \cap S \\
   &=   B \cap (A \cup A^c) \\
   &=   (B \cap A) \cup (B \cap A^c)
   \end{aligned}
   $$
   where $B \cap A$ and $B \cap A^c$ are mutually exclusive.
   \end{itemize}
   \end{block}
   \end{frame}

   %CODE
       % from matplotlib import rc
   % import pylab, numpy as np, sys
   % np.random.seed(0);import random; random.seed(0)
   % sys.path.append('/private/var/folders/dq/4_9bwd013ln6vvf_q110mwrh0000gn/T/tmpBBiJ_x')
   % f=pylab.gcf(); f.set_size_inches(6.0,6.0)
   % datadir ='/private/var/folders/dq/4_9bwd013ln6vvf_q110mwrh0000gn/T/tmpBBiJ_x/data'
   % import numpy as np, pylab, matplotlib
   % 
   % import matplotlib
   % rc('text', usetex=True)
   % cir = matplotlib.patches.Circle
   % 
   % a = pylab.gca()
   % # add a circle
   % E1 = cir((0.5,0.5), 0.4,ec="black", facecolor='yellow',lw=2, alpha=0.4)
   % E2 = cir((0.2,0.2), 0.4,ec="black", facecolor='blue',lw=2, alpha=0.4)
   % a.text(0.25,0.25, r'$A \cap B$', va='center', ha='center')
   % a.text(0.15,0.15, r'$A \cap B^c$', va='center', ha='center')
   % a.add_patch(E1)
   % a.add_patch(E2)
   % a.set_xticks([])
   % a.set_yticks([])
   % a.set_xlim([-0.3,1])
   % a.set_ylim([-0.3,1])
   % 


   \begin{frame}
   \frametitle{Non-mutually exclusive events}
   \begin{center}
   \resizebox{!}{2.7in}{\includegraphics{./images/inline/276c7f4660.pdf}}    
   \end{center}

   \end{frame}

   %%%%%%%%%%%%%%%%%%%%%%%%%%%%%%%%%%%%%%%%%%%%%%%%%%%%%%%%%%%%

   \begin{frame} \frametitle{Probability}

   \begin{block}
   {Example of using complements}
   \begin{itemize}
   \item For Box \# 2, if we draw with replacement, what are
   the chances it will take less than 5 draws to draw 1st blue marble?
   \item If $E$=\{takes less than 5 draws to draw 1st blue marble\},
   then
   $$
   \begin{aligned}
   E^c &=\{\text{takes 5 or more draws to draw 1st blue marble}\} \\
   &=\{\text{first 4 draws are red}\} \\
   \end{aligned}
   $$
   \item By independence, $
   P(\text{first 4 draws are red}) = \left(\frac{2}{5}\right)^4
   $
   \item Therefore, $P(\text{takes less than 5 draws to draw 1st blue marble}) = 1 -  \left(\frac{2}{5}\right)^4 = 97\%$
   \end{itemize}
   \end{block}
   \end{frame}

   %%%%%%%%%%%%%%%%%%%%%%%%%%%%%%%%%%%%%%%%%%%%%%%%%%%%%%%%%%%%

   \begin{frame} \frametitle{Probability}

   \begin{block}
   {Bayes' theorem}
   \begin{itemize}
   \item Given two events $A$ and $B$
   $$
   \begin{aligned}
   P(A|B) &= \frac{P(B \, \text{and} \,  A)}{P(B)} \\
   &= \frac{P(B|A)\times P(A)}{P(B)}
   \end{aligned}
   $$
   \item A consequence of the multiplication rule \dots

   \end{itemize}
   \end{block}
   \end{frame}

   %%%%%%%%%%%%%%%%%%%%%%%%%%%%%%%%%%%%%%%%%%%%%%%%%%%%%%%%%%%%

   \begin{frame} \frametitle{Probability}

   \begin{block}
   {Bayes' theorem}
   \begin{itemize}
   \item By the properties of complements
   $$
   \begin{aligned}
   P(B) &= P(B \cap A) + P(B \cap A^c) \\
   &=  P(B|A) \times P(A) + P(B|A^c) \times P(A^c)
   \end{aligned}
   $$
   \item Another version of Bayes' theorem
   $$
   \begin{aligned}
   P(A|B) &= \frac{P(B|A) \times P(A)}{P(B|A) \times P(A) + P(B|A^c) \times P(A^c)     } \\
   \end{aligned}
   $$

   \end{itemize}
   \end{block}
   \end{frame}

   %%%%%%%%%%%%%%%%%%%%%%%%%%%%%%%%%%%%%%%%%%%%%%%%%%%%%%%%%%%%

   \begin{frame} \frametitle{Probability}

   \begin{block}
   {Example of Bayes' theorem: drawing marbles without replacement from Box \# 2}
   \begin{itemize}
   \item Let
   $$
   \begin{aligned}
   A&=\{\text{draw a red marble on first draw}\} \\
   B&=\{\text{draw a blue marble on second draw}\} \\
   \end{aligned}
   $$
   Compute $P(A|B)$.
   \item  What do we know?
   $$
   \begin{aligned}
   P(A) &= \frac{2}{5} \\
   P(B|A) &= \frac{3}{4} \\
   \end{aligned}
   $$
   \item We need $P(B) = P(B|A) \times P(A) + P(B|A^c) \times P(A^c)$.
   \end{itemize}
   \end{block}
   \end{frame}

   %%%%%%%%%%%%%%%%%%%%%%%%%%%%%%%%%%%%%%%%%%%%%%%%%%%%%%%%%%%%

   \begin{frame} \frametitle{Probability}

   \begin{block}
   {Example (continued)}

   \begin{itemize}
   \item Note that $A^c=\{\text{draw a blue marble on first draw}\}$.
   \item We know
   $$
   \begin{aligned}
   P(A^c) &= \frac{3}{5} \\
   P(B|A^c) &= \frac{1}{2} \\
   \end{aligned}
   $$
   \item Therefore,
   $$
   \begin{aligned}
   P(B) &= \frac{3}{4} \times \frac{2}{5}  + \frac{1}{2} \times \frac{3}{5} = \frac{3}{10}     \\
   P(A|B) &= \frac{ \frac{3}{4} \times \frac{2}{5}}{\frac{3}{4} \times \frac{2}{5}  + \frac{1}{2} \times \frac{3}{5}} = \frac{1}{2}
   \end{aligned}
   $$
   \end{itemize}
   \end{block}
   \end{frame}

   %%%%%%%%%%%%%%%%%%%%%%%%%%%%%%%%%%%%%%%%%%%%%%%%%%%%%%%%%%%%

   \begin{frame} \frametitle{Probability}

   \begin{block}
   {A second example}

   \begin{itemize}
   \item Suppose a patient from some population is tested for a disease based on   some diagnostic test.
   \item The prevalence of the disease is 0.1\% in the population.
   \item If a patient has the disease, the test result is positive
   with probability 95 \%. ({\em True positive})
   \item If a patient does not have the disease, the test result is positive
   with probability 1 \%. ({\em False positive}).
   \item What is the probability a patient has the disease
   given a positive test result?
   \end{itemize}
   \end{block}
   \end{frame}

   %%%%%%%%%%%%%%%%%%%%%%%%%%%%%%%%%%%%%%%%%%%%%%%%%%%%%%%%%%%%

   \begin{frame} \frametitle{Probability}

   \begin{block}
   {A second example}

   \begin{itemize}
   \item Let
   $$
   \begin{aligned}
   D &= \{\text{patient has disease}\}     \\
   T^+ &= \{\text{test result is positive}\}     \\
   \end{aligned}
   $$
   \item We are given
   $$
   \begin{aligned}
   P(D) &= 0.001 \\
   P(T^+|D) &= 0.95 \\
   P(T^+|D^c) &= 0.01 \\
   \end{aligned}
   $$
   \item We want to compute $P(D|T^+)$.
   \end{itemize}
   \end{block}
   \end{frame}

   %%%%%%%%%%%%%%%%%%%%%%%%%%%%%%%%%%%%%%%%%%%%%%%%%%%%%%%%%%%%

   \begin{frame} \frametitle{Probability}

   \begin{block}
   {Example continued}

   \begin{itemize}
     \item By Bayes' theorem
   $$
   \begin{aligned}
   P(D|T^+) &= \frac{P(T^+|D) \times P(D)}{P(T^+|D) \times P(D) + P(T^+|D^c) \times P(D^c)} \\
   &= \frac{0.95 \times 0.001}{0.95 \times 0.001 + 0.01 \times 0.999} \\
   &= 8.7 \%
   \end{aligned}
   $$

   \item If the test makers improve their false positive rate to 0.001 then
   $$
   \begin{aligned}
   P(D|T^+)
   &= \frac{0.95 \times 0.001}{0.95 \times 0.001 + 0.001 \times 0.999} \\
   &= 48.7 \%
   \end{aligned}
   $$
   \end{itemize}
   \end{block}
   \end{frame}

   %%%%%%%%%%%%%%%%%%%%%%%%%%%%%%%%%%%%%%%%%%%%%%%%%%%%%%%%%%%%

   \begin{frame} 

   \end{frame}

   \end{document}
