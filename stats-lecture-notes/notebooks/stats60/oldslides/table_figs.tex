   \documentclass[handout]{beamer}



   \mode<presentation>
   {
     \usetheme{PaloAlto}
   \setbeamertemplate{footline}[page number]

     \setbeamercolor{sidebar}{bg=white, fg=black}
     \setbeamercolor{frametitle}{bg=white, fg=black}
     % or ...
     \setbeamercolor{logo}{bg=white}
     \setbeamercolor{block body}{parent=normal text,bg=white}
     \setbeamercolor{author in sidebar}{fg=black}
     \setbeamercolor{title in sidebar}{fg=black}


     \setbeamercolor*{block title}{use=structure,fg=structure.fg,bg=structure.fg!20!bg}
     \setbeamercolor*{block title alerted}{use=alerted text,fg=alerted text.fg,bg=alerted text.fg!20!bg}
     \setbeamercolor*{block title example}{use=example text,fg=example text.fg,bg=example text.fg!20!bg}


     \setbeamercolor{block body}{parent=normal text,use=block title,bg=block title.bg!50!bg}
     \setbeamercolor{block body alerted}{parent=normal text,use=block title alerted,bg=block title alerted.bg!50!bg}
     \setbeamercolor{block body example}{parent=normal text,use=block title example,bg=block title example.bg!50!bg}

     % or ...

     \setbeamercovered{transparent}
     % or whatever (possibly just delete it)
     \logo{\resizebox{!}{1.5cm}{\href{\basename{R}}{\includegraphics{image}}}}
   }

   \mode<handout>
   {
     \usetheme{PaloAlto}
     \usecolortheme{default}
     \setbeamercolor{sidebar}{bg=white, fg=black}
     \setbeamercolor{frametitle}{bg=white, fg=black}
     % or ...
     \setbeamercolor{logo}{bg=white}
     \setbeamercolor{block body}{parent=normal text,bg=white}
     \setbeamercolor{author in sidebar}{fg=black}
     \setbeamercolor{title in sidebar}{fg=black}
     \setbeamercovered{transparent}
     % or whatever (possibly just delete it)
     \logo{}
   }

   \usepackage{epsdice}
   \usepackage[latin1]{inputenc}
   \usepackage{graphics}
   \usepackage{amsmath,eepic,epic}

   \newcommand{\figslide}[3]{
   \begin{frame}
   \frametitle{#1}
     \begin{center}
     \resizebox{!}{2.7in}{\includegraphics{#2}}    
     \end{center}
   {#3}
   \end{frame}
   }

   \newcommand{\fighslide}[4]{
   \begin{frame}
   \frametitle{#1}
     \begin{center}
     \resizebox{!}{#4}{\includegraphics{#2}}    
     \end{center}
   {#3}
   \end{frame}
   }

   \newcommand{\figwref}[1]{
   \href{#1}{\tiny \tt #1}}

   \newcommand{\B}[1]{\beta_{#1}}
   \newcommand{\Bh}[1]{\widehat{\beta}_{#1}}
   \newcommand{\V}{\text{Var}}
   \newcommand{\Cov}{\text{Cov}}
   \newcommand{\Vh}{\widehat{\V}}
   \newcommand{\s}{\sigma}
   \newcommand{\sh}{\widehat{\sigma}}

   \newcommand{\argmax}[1]{\mathop{\text{argmax}}_{#1}}
   \newcommand{\argmin}[1]{\mathop{\text{argmin}}_{#1}}
   \newcommand{\Ee}{\mathbb{E}}
   \newcommand{\Pp}{\mathbb{P}}
   \newcommand{\real}{\mathbb{R}}
   \newcommand{\Ybar}{\overline{Y}}
   \newcommand{\Yh}{\widehat{Y}}
   \newcommand{\Xbar}{\overline{X}}
   \newcommand{\Tr}{\text{Tr}}


   \newcommand{\model}{{\cal M}}

   \newcommand{\figvskip}{-0.7in}
   \newcommand{\fighskip}{-0.3in}
   \newcommand{\figheight}{3.5in}

   \newcommand{\Rcode}[1]{{\bf \tt #1 }}
   \newcommand{\Rtcode}[1]{{\tiny \bf \tt #1 }}
   \newcommand{\Rscode}[1]{{\small \bf \tt #1 }}

   \newcommand{\RR}{{\tt R} \;}
   \newcommand{\basename}[1]{http://stats60.stanford.edu/#1}
   \newcommand{\dataname}[1]{\basename{data/#1}}
   \newcommand{\Rname}[1]{\basename{R/#1}}

   \newcommand{\mycolor}[1]{{\color{blue} #1}}
   \newcommand{\basehref}[2]{\href{\basename{#1}}{\mycolor{#2}}}
   \newcommand{\Rhref}[2]{\href{\basename{R/#1}}{\mycolor{#2}}}
   \newcommand{\datahref}[2]{\href{\dataname{#1}}{\mycolor{#2}}}
   \newcommand{\X}{\pmb{X}}
   \newcommand{\Y}{\pmb{Y}}
   \newcommand{\be}{\pmb{varepsilon}}
   \newcommand{\logit}{\text{logit}}


   \title{Statistics 60: Introduction to Statistical Methods}
   \subtitle{Some figures} 
   \author{}% {Jonathan Taylor \\
   %Department of Statistics \\
   %Stanford University
   %}


   \begin{document}

   \begin{frame}
   \titlepage
   \end{frame}

   %CODE
       % from matplotlib import rc
   % import pylab, numpy as np, sys
   % np.random.seed(0);import random; random.seed(0)
   % sys.path.append('/private/var/folders/dq/4_9bwd013ln6vvf_q110mwrh0000gn/T/tmp3p6GjD')
   % f=pylab.gcf(); f.set_size_inches(8.0,6.0)
   % datadir ='/private/var/folders/dq/4_9bwd013ln6vvf_q110mwrh0000gn/T/tmp3p6GjD/data'
   % import pylab, numpy as np
   % import scipy.stats
   % 
   % df = 6
   % 
   % # The t region
   % 
   % x = np.linspace(-4,4,101)
   % pylab.plot(x,scipy.stats.t.pdf(x, df)*100, linewidth=2, label=r'$T$, df=%d, 10%%' % df, color='black')
   % x2 = np.linspace(scipy.stats.t.isf(0.10,df),4, 101)
   % y2 = scipy.stats.t.pdf(x2, df)
   % xf, yf = pylab.poly_between(x2, 0*x2, y2*100)
   % pylab.fill(xf, yf, facecolor='gray', hatch='\\', alpha=0.5)
   % pylab.text(scipy.stats.t.isf(0.10,df), -3, r'$t$', size=20)
   % # pylab.gca().set_xlabel('standardized units')
   % # pylab.gca().set_ylabel('% per standardized unit')
   % # pylab.legend()
   % pylab.gca().set_yticks([])
   % pylab.gca().set_xticks([])
   % pylab.gca().set_xlim([-3,3])
   % 


   \begin{frame}
   \frametitle{Student's $T$}
   \begin{center}
   \resizebox{!}{2.7in}{\includegraphics{./images/inline/7f296ee3ec.pdf}}    
   \end{center}
   Comparison of two-sided {\color{blue} 5\% rejection rule}, df=4
   \end{frame}

   %CODE
       % from matplotlib import rc
   % import pylab, numpy as np, sys
   % np.random.seed(0);import random; random.seed(0)
   % sys.path.append('/private/var/folders/dq/4_9bwd013ln6vvf_q110mwrh0000gn/T/tmp3p6GjD')
   % f=pylab.gcf(); f.set_size_inches(8.0,6.0)
   % datadir ='/private/var/folders/dq/4_9bwd013ln6vvf_q110mwrh0000gn/T/tmp3p6GjD/data'
   % import pylab, numpy as np
   % import scipy.stats
   % 
   % # The z region
   % 
   % x = np.linspace(-4,4,101)
   % pylab.plot(x,scipy.stats.norm.pdf(x)*100, linewidth=2, label=r'$Z$', color='black')
   % q = scipy.stats.norm.isf(0.1)
   % x2 = np.linspace(-q, q, 201)
   % y2 = scipy.stats.norm.pdf(x2)
   % xf, yf = pylab.poly_between(x2, 0*x2, y2*100)
   % pylab.fill(xf, yf, facecolor='gray', hatch='\\', alpha=0.5)
   % pylab.text(-q, -3, r'$-z$', size=20)
   % pylab.text(q, -3, r'$z$', size=20)
   % # pylab.gca().set_xlabel('standardized units')
   % # pylab.gca().set_ylabel('% per standardized unit')
   % # pylab.legend()
   % pylab.gca().set_xlim([-3,3])
   % pylab.gca().set_yticks([])
   % pylab.gca().set_xticks([])
   % 
   % import pylab, numpy as np
   % import scipy.stats
   % 
   % # The X2 region
   % 
   % x = np.linspace(0,10,101)
   % pylab.plot(x,scipy.stats.norm.pdf(x)*100, linewidth=2, label=r'$Z$', color='black')
   % q = scipy.stats.norm.isf(0.1)
   % x2 = np.linspace(-q, q, 201)
   % y2 = scipy.stats.norm.pdf(x2)
   % xf, yf = pylab.poly_between(x2, 0*x2, y2*100)
   % pylab.fill(xf, yf, facecolor='gray', hatch='\\', alpha=0.5)
   % pylab.text(-q, -3, r'$-z$', size=20)
   % pylab.text(q, -3, r'$z$', size=20)
   % # pylab.gca().set_xlabel('standardized units')
   % # pylab.gca().set_ylabel('% per standardized unit')
   % # pylab.legend()
   % pylab.gca().set_xlim([-3,3])
   % pylab.gca().set_yticks([])
   % pylab.gca().set_xticks([])
   % 


   \begin{frame}
   \frametitle{Student's $T$}
   \begin{center}
   \resizebox{!}{2.7in}{\includegraphics{./images/inline/0799a218b9.pdf}}    
   \end{center}
   Comparison of two-sided {\color{blue} 5\% rejection rule}, df=4
   \end{frame}

   %CODE
       % from matplotlib import rc
   % import pylab, numpy as np, sys
   % np.random.seed(0);import random; random.seed(0)
   % sys.path.append('/private/var/folders/dq/4_9bwd013ln6vvf_q110mwrh0000gn/T/tmp3p6GjD')
   % f=pylab.gcf(); f.set_size_inches(8.0,6.0)
   % datadir ='/private/var/folders/dq/4_9bwd013ln6vvf_q110mwrh0000gn/T/tmp3p6GjD/data'
   % import pylab, numpy as np
   % import scipy.stats
   % df = 5
   % x = np.linspace(0,20,501)
   % y = scipy.stats.chi2.pdf(x, df)
   % q = scipy.stats.chi2.isf(0.25, df)
   % x2 = np.linspace(q,20,501)
   % y2 = scipy.stats.chi2.pdf(x2, df)
   % pylab.plot(x,y*100, linewidth=2, color='black')
   % xf, yf = pylab.poly_between(x2, 0*x2, y2*100)
   % pylab.fill(xf, yf, facecolor='gray', hatch='\\', alpha=0.5)
   % #pylab.gca().set_xlabel('standardized units')
   % #pylab.gca().set_ylabel('% per standardized unit')
   % pylab.gca().set_xticks([])
   % pylab.gca().set_yticks([])
   % pylab.gca().set_xlim([0,15])
   % 
   % pylab.text(q, -1.5, r'$\chi^2$', size=20)
   % 


   \begin{frame}
   \frametitle{$\chi^2$ curves}
   \begin{center}
   \resizebox{!}{2.7in}{\includegraphics{./images/inline/615b0d4e00.pdf}}    
   \end{center}
   The 5\% rejection rule for $\chi^2_5$
   \end{frame}

   %%%%%%%%%%%%%%%%%%%%%%%%%%%%%%%%%%%%%%%%%%%%%%%%%%%%%%%%%%%%

   \begin{frame} 

   \end{frame}

   \end{document}
