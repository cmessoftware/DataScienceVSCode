   \documentclass[handout]{beamer}



   \mode<presentation>
   {
     \usetheme{PaloAlto}
   \setbeamertemplate{footline}[page number]

     \setbeamercolor{sidebar}{bg=white, fg=black}
     \setbeamercolor{frametitle}{bg=white, fg=black}
     % or ...
     \setbeamercolor{logo}{bg=white}
     \setbeamercolor{block body}{parent=normal text,bg=white}
     \setbeamercolor{author in sidebar}{fg=black}
     \setbeamercolor{title in sidebar}{fg=black}


     \setbeamercolor*{block title}{use=structure,fg=structure.fg,bg=structure.fg!20!bg}
     \setbeamercolor*{block title alerted}{use=alerted text,fg=alerted text.fg,bg=alerted text.fg!20!bg}
     \setbeamercolor*{block title example}{use=example text,fg=example text.fg,bg=example text.fg!20!bg}


     \setbeamercolor{block body}{parent=normal text,use=block title,bg=block title.bg!50!bg}
     \setbeamercolor{block body alerted}{parent=normal text,use=block title alerted,bg=block title alerted.bg!50!bg}
     \setbeamercolor{block body example}{parent=normal text,use=block title example,bg=block title example.bg!50!bg}

     % or ...

     \setbeamercovered{transparent}
     % or whatever (possibly just delete it)
     \logo{\resizebox{!}{1.5cm}{\href{\basename{R}}{\includegraphics{image}}}}
   }

   \mode<handout>
   {
     \usetheme{PaloAlto}
     \usecolortheme{default}
     \setbeamercolor{sidebar}{bg=white, fg=black}
     \setbeamercolor{frametitle}{bg=white, fg=black}
     % or ...
     \setbeamercolor{logo}{bg=white}
     \setbeamercolor{block body}{parent=normal text,bg=white}
     \setbeamercolor{author in sidebar}{fg=black}
     \setbeamercolor{title in sidebar}{fg=black}
     \setbeamercovered{transparent}
     % or whatever (possibly just delete it)
     \logo{}
   }

   \usepackage{epsdice}
   \usepackage[latin1]{inputenc}
   \usepackage{graphics}
   \usepackage{amsmath,eepic,epic}

   \newcommand{\figslide}[3]{
   \begin{frame}
   \frametitle{#1}
     \begin{center}
     \resizebox{!}{2.7in}{\includegraphics{#2}}    
     \end{center}
   {#3}
   \end{frame}
   }

   \newcommand{\fighslide}[4]{
   \begin{frame}
   \frametitle{#1}
     \begin{center}
     \resizebox{!}{#4}{\includegraphics{#2}}    
     \end{center}
   {#3}
   \end{frame}
   }

   \newcommand{\figwref}[1]{
   \href{#1}{\tiny \tt #1}}

   \newcommand{\B}[1]{\beta_{#1}}
   \newcommand{\Bh}[1]{\widehat{\beta}_{#1}}
   \newcommand{\V}{\text{Var}}
   \newcommand{\Cov}{\text{Cov}}
   \newcommand{\Vh}{\widehat{\V}}
   \newcommand{\s}{\sigma}
   \newcommand{\sh}{\widehat{\sigma}}

   \newcommand{\argmax}[1]{\mathop{\text{argmax}}_{#1}}
   \newcommand{\argmin}[1]{\mathop{\text{argmin}}_{#1}}
   \newcommand{\Ee}{\mathbb{E}}
   \newcommand{\Pp}{\mathbb{P}}
   \newcommand{\real}{\mathbb{R}}
   \newcommand{\Ybar}{\overline{Y}}
   \newcommand{\Yh}{\widehat{Y}}
   \newcommand{\Xbar}{\overline{X}}
   \newcommand{\Tr}{\text{Tr}}


   \newcommand{\model}{{\cal M}}

   \newcommand{\figvskip}{-0.7in}
   \newcommand{\fighskip}{-0.3in}
   \newcommand{\figheight}{3.5in}

   \newcommand{\Rcode}[1]{{\bf \tt #1 }}
   \newcommand{\Rtcode}[1]{{\tiny \bf \tt #1 }}
   \newcommand{\Rscode}[1]{{\small \bf \tt #1 }}

   \newcommand{\RR}{{\tt R} \;}
   \newcommand{\basename}[1]{http://stats60.stanford.edu/#1}
   \newcommand{\dataname}[1]{\basename{data/#1}}
   \newcommand{\Rname}[1]{\basename{R/#1}}

   \newcommand{\mycolor}[1]{{\color{blue} #1}}
   \newcommand{\basehref}[2]{\href{\basename{#1}}{\mycolor{#2}}}
   \newcommand{\Rhref}[2]{\href{\basename{R/#1}}{\mycolor{#2}}}
   \newcommand{\datahref}[2]{\href{\dataname{#1}}{\mycolor{#2}}}
   \newcommand{\X}{\pmb{X}}
   \newcommand{\Y}{\pmb{Y}}
   \newcommand{\be}{\pmb{varepsilon}}
   \newcommand{\logit}{\text{logit}}


   \title{Statistics 60: Introduction to Statistical Methods}
   \subtitle{Chapter 27: More tests for averages} 
   \author{}% {Jonathan Taylor \\
   %Department of Statistics \\
   %Stanford University
   %}


   \begin{document}

   \begin{frame}
   \titlepage
   \end{frame}

   %%%%%%%%%%%%%%%%%%%%%%%%%%%%%%%%%%%%%%%%%%%%%%%%%%%%%%%%%%%%

   \begin{frame} \frametitle{Comparing two samples}

   \begin{block}
   {NAEP reading test}
   \begin{itemize}
   \item In 1990, sample average was 290.

   \item In 2004, sample average was 285.

   \item Is this chance variation, or has something changed?

   \end{itemize}
   \end{block}
   \end{frame}

   %%%%%%%%%%%%%%%%%%%%%%%%%%%%%%%%%%%%%%%%%%%%%%%%%%%%%%%%%%%%

   \begin{frame} \frametitle{Comparing two samples}

   \begin{block}
   {Two boxes}
   \begin{itemize}
   \item One box for 1990, average(1990 sample) = 290, SD(2004 sample) = 37.

   \item A second box for 2004, average(2004 sample) = 285, SD(2004 sample) = 40.

   \item So,
   $$
   \begin{aligned}
   \text{SE(average(1990 sample))} &\approx 37 / \sqrt{1000} \\
    &\approx 1.2\\
   \text{SE(average(2004 sample))} &\approx 40 / \sqrt{1000} \\
   &\approx 1.3\\
   \end{aligned}
   $$

   \end{itemize}
   \end{block}
   \end{frame}

   %%%%%%%%%%%%%%%%%%%%%%%%%%%%%%%%%%%%%%%%%%%%%%%%%%%%%%%%%%%%

   \begin{frame} \frametitle{Comparing two samples}

   \begin{block}
   {Two boxes}
   \begin{itemize}
   \item The null hypothesis is: {\color{blue} ``the average of the 1990 equals   (or is less than or equal to) the
   average of the 2004 box.''}

   \item The alternative is {\color{blue} ``the average of the 1990 box is greater than the  average of the 2004 box.''}

   \item Our {\color{orange} $z$-score} will have ${\color{orange}285-290}-{\color{blue}{0}}={\color{orange} -5}$ in the numerator. We reject for large ${\color{orange} |z||}$ values that are negative.

   \item What about the denominator?

   \item We are using
   $$
   {\color{orange} \text{average(2004 sample) - average(1990 sample)}}
   $$
   to estimate the difference between the average
   of the 2004 and 1990 box averages.


   \end{itemize}
   \end{block}
   \end{frame}

   %%%%%%%%%%%%%%%%%%%%%%%%%%%%%%%%%%%%%%%%%%%%%%%%%%%%%%%%%%%%

   \begin{frame} \frametitle{Comparing two samples}

   \begin{block}
   {SE of a difference}
   \begin{itemize}



   \item So, the denominator should be
   $$
   {\color{blue} \text{SE}}({\color{orange} \text{average(2004 sample) -
   average(1990 sample)}}).
   $$

   \item The SE of the difference of two {\bf independent}
   quantities can be found from the individual SEs.

   \item For our two box model, the {\color{orange} average(1990 sample)}
   is independent of the {\color{orange} average(2004 sample)}.

   \item The rule is
   $$
   \begin{aligned}
   \lefteqn{{\color{blue} \text{SE}}({\color{orange}
   \text{average(2004 sample) -
   average(1990 sample)}})} \\
   & \qquad \quad = \sqrt{{\color{blue}\text{SE}}({\color{orange} \text{1990 sample}})^2 + {\color{blue}\text{SE}}({\color{orange} \text{2004 sample}})^2}.
   \end{aligned}
   $$


   \end{itemize}
   \end{block}
   \end{frame}

   %%%%%%%%%%%%%%%%%%%%%%%%%%%%%%%%%%%%%%%%%%%%%%%%%%%%%%%%%%%%

   \begin{frame} \frametitle{Comparing two samples}

   \begin{block}
   {Example (continued)}
   \begin{itemize}

   \item Applying the rule,
   $$
   \begin{aligned}
   \lefteqn{ {\color{blue} \text{SE}}({\color{orange}
   \text{average(2004 sample)} -}} \\
   & \qquad \quad - \text{\color{orange}  average(1990 sample)})  \\
   & \qquad \approx \sqrt{1.2^2+1.3^2} \approx 1.8
   \end{aligned}
   $$
   \item The $z$-statistic is
   $$
   {\color{orange} z} = \frac{-5}{1.8} \approx -2.8
   $$

   \item The one-sided $P$-value is about $0.3\%$.

   \end{itemize}
   \end{block}
   \end{frame}

   %%%%%%%%%%%%%%%%%%%%%%%%%%%%%%%%%%%%%%%%%%%%%%%%%%%%%%%%%%%%

   \begin{frame} \frametitle{Comparing two samples}

   \begin{block}
   {Example}
   \begin{itemize}
   \item In 1999, 13\% of the 17 year-old students had taken calculus
   compared to 17\% in 2004 according to the NAEP samples.
   Is the difference real or chance variation?

   \item This question asks us to compare two proportions. We can take
   the null hypothesis to be ``the proportion of 17 year olds who took
   calculus in 1999 is equal to (or greater than or equal to) the
   proportion who took calculus in 2004.''

   \item The alternative is ``the proportion of 17 year olds
   who took calculus in 1999 is less than the proportion who took calculus
   in 2004.''
   \end{itemize}
   \end{block}
   \end{frame}

   %%%%%%%%%%%%%%%%%%%%%%%%%%%%%%%%%%%%%%%%%%%%%%%%%%%%%%%%%%%%

   \begin{frame} \frametitle{Comparing two samples}

   \begin{block}
   {Example}
   \begin{itemize}
   \item We could rewrite this as $H_0: p_{1999} \geq p_{2004}$ vs.
   $H_a:p_{1999} < p_{2004}$.

   \item Our sample estimates are $\widehat{p}_{1999}=13\%, \widehat{p}_{2004}=17\%$.

   \item They have estimated standard errors: $\text{SE}(\widehat{p}_{1999}) \approx \sqrt{0.13 \times 0.87 / 1000} \approx 1.1\%$, $\text{SE}(\widehat{p}_{2004}) \approx \sqrt{0.17 \times 0.83 / 1000} = 1.2\%$,

   \item The \text{SE} of the difference is
   $$
   \text{SE}(\widehat{p}_{1999} - \widehat{p}_{2004}) \approx \sqrt{1.1^2 + 1.2^2} \approx 1.6\%.
   $$

   \end{itemize}
   \end{block}
   \end{frame}

   %%%%%%%%%%%%%%%%%%%%%%%%%%%%%%%%%%%%%%%%%%%%%%%%%%%%%%%%%%%%

   \begin{frame} \frametitle{Comparing two samples}

   \begin{block}
   {Example}
   \begin{itemize}
   \item The $z$-score is
   $$
   z = \frac{13 - 17 - 0}{1.6} \approx -2.5
   $$

   \item The one-sided $P$-value is about 1\%.

   \end{itemize}
   \end{block}
   \end{frame}

   %%%%%%%%%%%%%%%%%%%%%%%%%%%%%%%%%%%%%%%%%%%%%%%%%%%%%%%%%%%%

   \begin{frame} \frametitle{Randomized experiments}

   \begin{block}
   {Randomized experiments}
   \begin{itemize}
   \item This {\em two-sample $z$-test} can also be used
   for randomized controlled experiments.

   \item Example:  200 subjects are split randomly into treatment
   and placebo in a study on vitamin C on the number of colds.

   \item In the treatment group average(treatment group)=2.3,
   SD(treatment group) = 3.1.

   \item In the placebo group average(placebo group)=2.6,
   SD(placebo group) = 2.9.

   \item Is there a difference in the number of colds in treatment
   vs. placebo group?


   \end{itemize}
   \end{block}
   \end{frame}

   %%%%%%%%%%%%%%%%%%%%%%%%%%%%%%%%%%%%%%%%%%%%%%%%%%%%%%%%%%%%

   \begin{frame} \frametitle{Randomized experiments}

   \begin{block}
   {Example (continued)}
   \begin{itemize}
   \item Naively applying the two sample
   $z$-test to this situation, yields
   $$
   z = \frac{(2.3 - 2.6) - 0}{\sqrt{\left(\frac{3.1}{\sqrt{100}}\right)^2 + \left(\frac{2.9}{\sqrt{100}}\right)^2}} = \frac{-0.3}{0.42} \approx -0.7.
   $$


   \item If we had taken a one-sided alternative, this would be a $P$-value of about 25\%.

   \item Our two-box model does not apply here because the groups were selected
   from the same ``box'' of 200 subjects.

   \item The two samples are not independent: if a patient is in treatment,
   he/she cannot also be in the control group.

   \end{itemize}
   \end{block}
   \end{frame}

   %%%%%%%%%%%%%%%%%%%%%%%%%%%%%%%%%%%%%%%%%%%%%%%%%%%%%%%%%%%%

   \begin{frame} \frametitle{Randomized experiments}

   \begin{block}
   {Box model for experiment}
   \begin{itemize}
   \item The best box model for this randomized experiment is a box
   with 200 tickets.

   \item Each ticket has two responses for each subject:
   \begin{itemize}
   \item ${\color{green} A}$: the number of colds if they receive the vitamin C;
   \item ${\color{red} B}$: the number of colds if they receive the placebo.
   \end{itemize}

   \item In the experiment, we only observe ${\color{green} A}$ or
   ${\color{red} B}$ for each subject.

   \item Which one we see depends on the randomization.

   \item Statistical theory says the $z$-test is applicable if the
   sample is large enough to test
   ${\color{blue} H_0: \text{average of all {\color{green} $A$}'s $\geq$ average of all {\color{red} $B$}'s}}$
    against ${\color{blue} H_a: \text{average of all {\color{green} $A$}'s $<$
    average of all {\color{red} $B$}'s}}$
   \end{itemize}
   \end{block}
   \end{frame}

   %CODE
       % from matplotlib import rc
   % import pylab, numpy as np, sys
   % np.random.seed(0);import random; random.seed(0)
   % sys.path.append('/private/var/folders/dq/4_9bwd013ln6vvf_q110mwrh0000gn/T/tmpMRJ55K')
   % f=pylab.gcf(); f.set_size_inches(8.0,6.0)
   % datadir ='/private/var/folders/dq/4_9bwd013ln6vvf_q110mwrh0000gn/T/tmpMRJ55K/data'
   % import numpy as np, pylab
   % import random
   % 
   % ## sample2 = [np.random.randint(0,6) for _  in range(200)]
   % ## sample = [np.random.randint(0,6) for _  in range(200)]
   % ## X = np.array([(s1,s2) for s1, s2 in zip(sample, sample2)], np.dtype([('sample1', np.int), ('sample2', np.int)]))
   % ## treatment = random.sample(range(200), 100)
   % ## treatment_bool = np.zeros(X.shape, np.bool)
   % ## for i in treatment:
   % ##     treatment_bool[i] = True
   % ## X = pylab.rec_append_fields(X, 'treatment', treatment_bool)
   % ## pylab.rec2csv(X, '%s/twosample.csv' % datadir)
   % 
   % import random
   % X = np.mgrid[0:10:10j,0:20:20j].reshape((2,200))   + np.random.sample((2,200)) * 0.05
   % X = X.T
   % sample = pylab.csv2rec('%s/twosample.csv' % datadir)
   % 
   % treatment = sample['treatment'].astype(np.bool)
   % treatment_sample = sample[treatment]['sample1']
   % placebo_sample = sample[~treatment]['sample2']
   % 
   % 
   % for i in range(200):
   %    pylab.fill_between([X[i,0]-.4,X[i,0]], [X[i,1]-.4,X[i,1]-.4], [X[i,1]+.3,X[i,1]+.3], facecolor='green', alpha=0.3)
   %    pylab.fill_between([X[i,0],X[i,0]+.4], [X[i,1]-.4,X[i,1]-.4], [X[i,1]+.3,X[i,1]+.3], facecolor='red', alpha=0.3)
   %    pylab.text(X[i,0]-0.2, X[i,1], '%d' % sample['sample1'][i], ha='center', va='center', size=10)
   %    pylab.text(X[i,0]+0.2, X[i,1], '%d' % sample['sample2'][i], ha='center', va='center', size=10)
   %    pylab.gca().set_xticks([]);    pylab.gca().set_xlim([-1,11])
   %    pylab.gca().set_yticks([]);    pylab.gca().set_ylim([-1,21])
   % 


   \begin{frame}
   \frametitle{Box model (hypothetical data)}
   \begin{center}
   \resizebox{!}{2.7in}{\includegraphics{./images/inline/5223837e44.pdf}}    
   \end{center}
   The complete data (unobserved)
   \end{frame}

   %CODE
       % from matplotlib import rc
   % import pylab, numpy as np, sys
   % np.random.seed(0);import random; random.seed(0)
   % sys.path.append('/private/var/folders/dq/4_9bwd013ln6vvf_q110mwrh0000gn/T/tmpMRJ55K')
   % f=pylab.gcf(); f.set_size_inches(8.0,6.0)
   % datadir ='/private/var/folders/dq/4_9bwd013ln6vvf_q110mwrh0000gn/T/tmpMRJ55K/data'
   % import numpy as np, pylab
   % import random
   % 
   % ## sample2 = [np.random.randint(0,6) for _  in range(200)]
   % ## sample = [np.random.randint(0,6) for _  in range(200)]
   % ## X = np.array([(s1,s2) for s1, s2 in zip(sample, sample2)], np.dtype([('sample1', np.int), ('sample2', np.int)]))
   % ## treatment = random.sample(range(200), 100)
   % ## treatment_bool = np.zeros(X.shape, np.bool)
   % ## for i in treatment:
   % ##     treatment_bool[i] = True
   % ## X = pylab.rec_append_fields(X, 'treatment', treatment_bool)
   % ## pylab.rec2csv(X, '%s/twosample.csv' % datadir)
   % 
   % import random
   % X = np.mgrid[0:10:10j,0:20:20j].reshape((2,200))   + np.random.sample((2,200)) * 0.05
   % X = X.T
   % sample = pylab.csv2rec('%s/twosample.csv' % datadir)
   % 
   % treatment = sample['treatment'].astype(np.bool)
   % treatment_sample = sample[treatment]['sample1']
   % placebo_sample = sample[~treatment]['sample2']
   % 
   % 
   % for i in range(200):
   %    if treatment[i]:
   %        pylab.fill_between([X[i,0]-.4,X[i,0]], [X[i,1]-.4,X[i,1]-.4], [X[i,1]+.3,X[i,1]+.3], facecolor='green', alpha=0.3)
   %        pylab.fill_between([X[i,0],X[i,0]+.4], [X[i,1]-.4,X[i,1]-.4], [X[i,1]+.3,X[i,1]+.3], facecolor='red', alpha=0.3)
   %        pylab.text(X[i,0]-0.2, X[i,1], '%d' % sample['sample1'][i], ha='center', va='center', size=10)
   % pylab.gca().set_xticks([]);    pylab.gca().set_xlim([-1,11])
   % pylab.gca().set_yticks([]);    pylab.gca().set_ylim([-1,21])
   % pylab.title("average(treatment)=%0.1f, SD(treatment)%0.1f" % (treatment_sample.mean(), treatment_sample.std()))
   % 


   \begin{frame}
   \frametitle{Box model (hypothetical data)}
   \begin{center}
   \resizebox{!}{2.7in}{\includegraphics{./images/inline/c8f29b844c.pdf}}    
   \end{center}
   The treatment sample (observed).
   \end{frame}

   %CODE
       % from matplotlib import rc
   % import pylab, numpy as np, sys
   % np.random.seed(0);import random; random.seed(0)
   % sys.path.append('/private/var/folders/dq/4_9bwd013ln6vvf_q110mwrh0000gn/T/tmpMRJ55K')
   % f=pylab.gcf(); f.set_size_inches(8.0,6.0)
   % datadir ='/private/var/folders/dq/4_9bwd013ln6vvf_q110mwrh0000gn/T/tmpMRJ55K/data'
   % import numpy as np, pylab
   % import random
   % 
   % ## sample2 = [np.random.randint(0,6) for _  in range(200)]
   % ## sample = [np.random.randint(0,6) for _  in range(200)]
   % ## X = np.array([(s1,s2) for s1, s2 in zip(sample, sample2)], np.dtype([('sample1', np.int), ('sample2', np.int)]))
   % ## treatment = random.sample(range(200), 100)
   % ## treatment_bool = np.zeros(X.shape, np.bool)
   % ## for i in treatment:
   % ##     treatment_bool[i] = True
   % ## X = pylab.rec_append_fields(X, 'treatment', treatment_bool)
   % ## pylab.rec2csv(X, '%s/twosample.csv' % datadir)
   % 
   % import random
   % X = np.mgrid[0:10:10j,0:20:20j].reshape((2,200))   + np.random.sample((2,200)) * 0.05
   % X = X.T
   % sample = pylab.csv2rec('%s/twosample.csv' % datadir)
   % 
   % treatment = sample['treatment'].astype(np.bool)
   % treatment_sample = sample[treatment]['sample1']
   % placebo_sample = sample[~treatment]['sample2']
   % 
   % 
   % for i in range(200):
   %    if not treatment[i]:
   %        pylab.fill_between([X[i,0]-.4,X[i,0]], [X[i,1]-.4,X[i,1]-.4], [X[i,1]+.3,X[i,1]+.3], facecolor='green', alpha=0.3)
   %        pylab.fill_between([X[i,0],X[i,0]+.4], [X[i,1]-.4,X[i,1]-.4], [X[i,1]+.3,X[i,1]+.3], facecolor='red', alpha=0.3)
   %        pylab.text(X[i,0]+0.2, X[i,1], '%d' % sample['sample2'][i], ha='center', va='center', size=10)
   % pylab.gca().set_xticks([]);    pylab.gca().set_xlim([-1,11])
   % pylab.gca().set_yticks([]);    pylab.gca().set_ylim([-1,21])
   % pylab.title("average(placebo)=%0.1f, SD(placebo)%0.1f" % (placebo_sample.mean(), placebo_sample.std()))
   % 


   \begin{frame}
   \frametitle{Box model (hypothetical data)}
   \begin{center}
   \resizebox{!}{2.7in}{\includegraphics{./images/inline/13c9fa1bbf.pdf}}    
   \end{center}
   The placebo sample (observed).
   \end{frame}

   %CODE
       % from matplotlib import rc
   % import pylab, numpy as np, sys
   % np.random.seed(0);import random; random.seed(0)
   % sys.path.append('/private/var/folders/dq/4_9bwd013ln6vvf_q110mwrh0000gn/T/tmpMRJ55K')
   % f=pylab.gcf(); f.set_size_inches(8.0,6.0)
   % datadir ='/private/var/folders/dq/4_9bwd013ln6vvf_q110mwrh0000gn/T/tmpMRJ55K/data'
   % import numpy as np, pylab
   % import random
   % 
   % ## sample2 = [np.random.randint(0,6) for _  in range(200)]
   % ## sample = [np.random.randint(0,6) for _  in range(200)]
   % ## X = np.array([(s1,s2) for s1, s2 in zip(sample, sample2)], np.dtype([('sample1', np.int), ('sample2', np.int)]))
   % ## treatment = random.sample(range(200), 100)
   % ## treatment_bool = np.zeros(X.shape, np.bool)
   % ## for i in treatment:
   % ##     treatment_bool[i] = True
   % ## X = pylab.rec_append_fields(X, 'treatment', treatment_bool)
   % ## pylab.rec2csv(X, '%s/twosample.csv' % datadir)
   % 
   % import random
   % X = np.mgrid[0:10:10j,0:20:20j].reshape((2,200))   + np.random.sample((2,200)) * 0.05
   % X = X.T
   % sample = pylab.csv2rec('%s/twosample.csv' % datadir)
   % 
   % treatment = sample['treatment'].astype(np.bool)
   % treatment_sample = sample[treatment]['sample1']
   % placebo_sample = sample[~treatment]['sample2']
   % 
   % 
   % for i in range(200):
   %    pylab.fill_between([X[i,0]-.4,X[i,0]], [X[i,1]-.4,X[i,1]-.4], [X[i,1]+.3,X[i,1]+.3], facecolor='green', alpha=0.3)
   %    pylab.fill_between([X[i,0],X[i,0]+.4], [X[i,1]-.4,X[i,1]-.4], [X[i,1]+.3,X[i,1]+.3], facecolor='red', alpha=0.3)
   %    if treatment[i]:
   %        pylab.text(X[i,0]-0.2, X[i,1], '%d' % sample['sample1'][i], ha='center', va='center', size=10)
   %    elif not treatment[i]:
   %        pylab.text(X[i,0]+0.2, X[i,1], '%d' % sample['sample2'][i], ha='center', va='center', size=10)
   % pylab.gca().set_xticks([]);    pylab.gca().set_xlim([-1,11])
   % pylab.gca().set_yticks([]);    pylab.gca().set_ylim([-1,21])
   % 


   \begin{frame}
   \frametitle{Box model (hypothetical data)}
   \begin{center}
   \resizebox{!}{2.7in}{\includegraphics{./images/inline/e6d8032720.pdf}}    
   \end{center}
   The entire sample (observed).
   \end{frame}

   %%%%%%%%%%%%%%%%%%%%%%%%%%%%%%%%%%%%%%%%%%%%%%%%%%%%%%%%%%%%

   \begin{frame} \frametitle{Box model (hypothetical data)}

   \begin{block}
   {Example (continued)}

   \begin{itemize}
   \item We know
     $$
     \begin{aligned}
     \text{average(treatment)} &= 2.3 \\
     \text{SE(average(treatment))} &= 1.6 / \sqrt{100} = 0.16 \\
     \text{average(placebo)} &= 2.5 \\
     \text{SE(average(placebo))} &= 1.7 / \sqrt{100} = 0.17 \\
     \end{aligned}
     $$

   \end{itemize}
   \end{block}
   \end{frame}

   %%%%%%%%%%%%%%%%%%%%%%%%%%%%%%%%%%%%%%%%%%%%%%%%%%%%%%%%%%%%

   \begin{frame} \frametitle{Box model (hypothetical data)}

   \begin{block}
   {Example (continued)}

   \begin{itemize}


   \item Therefore,
     $$
     \begin{aligned}
     \lefteqn{\text{SE(average(treatment)-average(placebo))}} \\
     & \qquad = \sqrt{0.16^2+0.17^2} \\
     &= 0.23
     \end{aligned}
     $$
     and
     $$
     z = \frac{2.3-2.5}{0.23} = -0.9
     $$
     \item The one-sided $P$-value is 18\%, when we test
     $H_0: \text{average({\color{green} all treatments})} \geq \text{average({\color{red} all placebos})}$.
   \end{itemize}
   \end{block}
   \end{frame}

   %%%%%%%%%%%%%%%%%%%%%%%%%%%%%%%%%%%%%%%%%%%%%%%%%%%%%%%%%%%%

   \begin{frame} \frametitle{Randomized experiments}

   \begin{block}
   {Binary responses}
   \begin{itemize}
   \item The randomization model is not applicable only to
   quantitative things like the number of colds. The tickets
   can have 0-1 values.

   \item Suppose the experiment tests an experimental cancer treatment
   and our outcome is whether or not the subject is remission free 5 years
   after treatment.

   \item Each subject has a {\color{green} 1 or 0} for their
   outcome if they are treated, and a {\color{red} 1 or 0} for their
   outcome if they receive placebo (or, more likely, standard of care).

   \end{itemize}
   \end{block}
   \end{frame}

   %%%%%%%%%%%%%%%%%%%%%%%%%%%%%%%%%%%%%%%%%%%%%%%%%%%%%%%%%%%%

   \begin{frame} \frametitle{Randomized experiments}

   \begin{block}
   {Binary responses}
   \begin{itemize}
   \item In the experiment, we only observe ${\color{green} A}$ or
   ${\color{red} B}$ for each subject.

   \item Which one we see depends on the randomization.

   \item Statistical theory says the $z$-test is applicable if the
   sample is large enough to test
   ${\color{blue} H_0: \text{proportion of all {\color{green} 1}'s $\leq$ proportion of all  {\color{red} 1}'s}}$ against
      ${\color{blue} H_a: \text{proportion of all {\color{green} 1}'s $>$ proportion of all  {\color{red} 1}'s}}$.
   \end{itemize}
   \end{block}
   \end{frame}

   %CODE
       % from matplotlib import rc
   % import pylab, numpy as np, sys
   % np.random.seed(0);import random; random.seed(0)
   % sys.path.append('/private/var/folders/dq/4_9bwd013ln6vvf_q110mwrh0000gn/T/tmpMRJ55K')
   % f=pylab.gcf(); f.set_size_inches(8.0,6.0)
   % datadir ='/private/var/folders/dq/4_9bwd013ln6vvf_q110mwrh0000gn/T/tmpMRJ55K/data'
   % ## import random
   % ## sample2 = np.random.binomial(1, 0.6, (200,))
   % ## sample = np.random.binomial(1, 0.8, (200,))
   % ## X = np.array([(s1,s2) for s1, s2 in zip(sample, sample2)], np.dtype([('sample1', np.int), ('sample2', np.int)]))
   % ## treatment = random.sample(range(200), 100)
   % ## treatment_bool = np.zeros(X.shape, np.bool)
   % ## for i in treatment:
   % ##     treatment_bool[i] = True
   % ## X = pylab.rec_append_fields(X, 'treatment', treatment_bool)
   % ## pylab.rec2csv(X, 'data/twosample_binary.csv')
   % 
   % X = np.mgrid[0:10:10j,0:20:20j].reshape((2,200))   + np.random.sample((2,200)) * 0.05
   % X = X.T
   % sample = pylab.csv2rec('%s/twosample_binary.csv' % datadir)
   % 
   % treatment = sample['treatment'].astype(np.bool)
   % treatment_sample = sample[treatment]['sample1']
   % placebo_sample = sample[~treatment]['sample2']
   % 
   % 
   % for i in range(200):
   %    pylab.fill_between([X[i,0]-.4,X[i,0]], [X[i,1]-.4,X[i,1]-.4], [X[i,1]+.3,X[i,1]+.3], facecolor='green', alpha=0.3)
   %    pylab.fill_between([X[i,0],X[i,0]+.4], [X[i,1]-.4,X[i,1]-.4], [X[i,1]+.3,X[i,1]+.3], facecolor='red', alpha=0.3)
   %    pylab.text(X[i,0]-0.2, X[i,1], '%d' % sample['sample1'][i], ha='center', va='center', size=10)
   %    pylab.text(X[i,0]+0.2, X[i,1], '%d' % sample['sample2'][i], ha='center', va='center', size=10)
   %    pylab.gca().set_xticks([]);    pylab.gca().set_xlim([-1,11])
   %    pylab.gca().set_yticks([]);    pylab.gca().set_ylim([-1,21])
   % 


   \begin{frame}
   \frametitle{Box model (for binary outcome)}
   \begin{center}
   \resizebox{!}{2.7in}{\includegraphics{./images/inline/c4200da009.pdf}}    
   \end{center}

   \end{frame}

   %CODE
       % from matplotlib import rc
   % import pylab, numpy as np, sys
   % np.random.seed(0);import random; random.seed(0)
   % sys.path.append('/private/var/folders/dq/4_9bwd013ln6vvf_q110mwrh0000gn/T/tmpMRJ55K')
   % f=pylab.gcf(); f.set_size_inches(8.0,6.0)
   % datadir ='/private/var/folders/dq/4_9bwd013ln6vvf_q110mwrh0000gn/T/tmpMRJ55K/data'
   % ## import random
   % ## sample2 = np.random.binomial(1, 0.6, (200,))
   % ## sample = np.random.binomial(1, 0.8, (200,))
   % ## X = np.array([(s1,s2) for s1, s2 in zip(sample, sample2)], np.dtype([('sample1', np.int), ('sample2', np.int)]))
   % ## treatment = random.sample(range(200), 100)
   % ## treatment_bool = np.zeros(X.shape, np.bool)
   % ## for i in treatment:
   % ##     treatment_bool[i] = True
   % ## X = pylab.rec_append_fields(X, 'treatment', treatment_bool)
   % ## pylab.rec2csv(X, 'data/twosample_binary.csv')
   % 
   % X = np.mgrid[0:10:10j,0:20:20j].reshape((2,200))   + np.random.sample((2,200)) * 0.05
   % X = X.T
   % sample = pylab.csv2rec('%s/twosample_binary.csv' % datadir)
   % 
   % treatment = sample['treatment'].astype(np.bool)
   % treatment_sample = sample[treatment]['sample1']
   % placebo_sample = sample[~treatment]['sample2']
   % 
   % 
   % for i in range(200):
   %    if treatment[i]:
   %        pylab.fill_between([X[i,0]-.4,X[i,0]], [X[i,1]-.4,X[i,1]-.4], [X[i,1]+.3,X[i,1]+.3], facecolor='green', alpha=0.3)
   %        pylab.fill_between([X[i,0],X[i,0]+.4], [X[i,1]-.4,X[i,1]-.4], [X[i,1]+.3,X[i,1]+.3], facecolor='red', alpha=0.3)
   %        pylab.text(X[i,0]-0.2, X[i,1], '%d' % sample['sample1'][i], ha='center', va='center', size=10)
   %    pylab.gca().set_xticks([]);    pylab.gca().set_xlim([-1,11])
   %    pylab.gca().set_yticks([]);    pylab.gca().set_ylim([-1,21])
   % 
   % pylab.title("proportion(treatment)=%0.2f" % treatment_sample.mean())
   % 


   \begin{frame}
   \frametitle{Box model (for binary outcome)}
   \begin{center}
   \resizebox{!}{2.7in}{\includegraphics{./images/inline/06103b3884.pdf}}    
   \end{center}
   Treatment group (observed).
   \end{frame}

   %CODE
       % from matplotlib import rc
   % import pylab, numpy as np, sys
   % np.random.seed(0);import random; random.seed(0)
   % sys.path.append('/private/var/folders/dq/4_9bwd013ln6vvf_q110mwrh0000gn/T/tmpMRJ55K')
   % f=pylab.gcf(); f.set_size_inches(8.0,6.0)
   % datadir ='/private/var/folders/dq/4_9bwd013ln6vvf_q110mwrh0000gn/T/tmpMRJ55K/data'
   % # import random
   % ## sample2 = np.random.binomial(1, 0.6, (200,))
   % ## sample = np.random.binomial(1, 0.8, (200,))
   % ## X = np.array([(s1,s2) for s1, s2 in zip(sample, sample2)], np.dtype([('sample1', np.int), ('sample2', np.int)]))
   % ## treatment = random.sample(range(200), 100)
   % ## treatment_bool = np.zeros(X.shape, np.bool)
   % ## for i in treatment:
   % ##     treatment_bool[i] = True
   % ## X = pylab.rec_append_fields(X, 'treatment', treatment_bool)
   % ## pylab.rec2csv(X, 'data/twosample_binary.csv')
   % 
   % X = np.mgrid[0:10:10j,0:20:20j].reshape((2,200))   + np.random.sample((2,200)) * 0.05
   % X = X.T
   % sample = pylab.csv2rec('%s/twosample_binary.csv' % datadir)
   % 
   % treatment = sample['treatment'].astype(np.bool)
   % treatment_sample = sample[treatment]['sample1']
   % placebo_sample = sample[~treatment]['sample2']
   % 
   % 
   % for i in range(200):
   %    if treatment[i]:
   %        pylab.fill_between([X[i,0]-.4,X[i,0]], [X[i,1]-.4,X[i,1]-.4], [X[i,1]+.3,X[i,1]+.3], facecolor='green', alpha=0.3)
   %        pylab.fill_between([X[i,0],X[i,0]+.4], [X[i,1]-.4,X[i,1]-.4], [X[i,1]+.3,X[i,1]+.3], facecolor='red', alpha=0.3)
   %        pylab.text(X[i,0]+0.2, X[i,1], '%d' % sample['sample2'][i], ha='center', va='center', size=10)
   %    pylab.gca().set_xticks([]);    pylab.gca().set_xlim([-1,11])
   %    pylab.gca().set_yticks([]);    pylab.gca().set_ylim([-1,21])
   % 
   % pylab.title("proportion(standard of care)=%0.2f" % placebo_sample.mean())
   % 


   \begin{frame}
   \frametitle{Box model (for binary outcome)}
   \begin{center}
   \resizebox{!}{2.7in}{\includegraphics{./images/inline/65b0e43d22.pdf}}    
   \end{center}
   Standard of care group (observed).
   \end{frame}

   %CODE
       % from matplotlib import rc
   % import pylab, numpy as np, sys
   % np.random.seed(0);import random; random.seed(0)
   % sys.path.append('/private/var/folders/dq/4_9bwd013ln6vvf_q110mwrh0000gn/T/tmpMRJ55K')
   % f=pylab.gcf(); f.set_size_inches(8.0,6.0)
   % datadir ='/private/var/folders/dq/4_9bwd013ln6vvf_q110mwrh0000gn/T/tmpMRJ55K/data'
   % # import random
   % ## sample2 = np.random.binomial(1, 0.6, (200,))
   % ## sample = np.random.binomial(1, 0.8, (200,))
   % ## X = np.array([(s1,s2) for s1, s2 in zip(sample, sample2)], np.dtype([('sample1', np.int), ('sample2', np.int)]))
   % ## treatment = random.sample(range(200), 100)
   % ## treatment_bool = np.zeros(X.shape, np.bool)
   % ## for i in treatment:
   % ##     treatment_bool[i] = True
   % ## X = pylab.rec_append_fields(X, 'treatment', treatment_bool)
   % ## pylab.rec2csv(X, 'data/twosample_binary.csv')
   % 
   % X = np.mgrid[0:10:10j,0:20:20j].reshape((2,200))   + np.random.sample((2,200)) * 0.05
   % X = X.T
   % sample = pylab.csv2rec('%s/twosample_binary.csv' % datadir)
   % 
   % treatment = sample['treatment'].astype(np.bool)
   % treatment_sample = sample[treatment]['sample1']
   % placebo_sample = sample[~treatment]['sample2']
   % 
   % 
   % for i in range(200):
   %    pylab.fill_between([X[i,0]-.4,X[i,0]], [X[i,1]-.4,X[i,1]-.4], [X[i,1]+.3,X[i,1]+.3], facecolor='green', alpha=0.3)
   %    pylab.fill_between([X[i,0],X[i,0]+.4], [X[i,1]-.4,X[i,1]-.4], [X[i,1]+.3,X[i,1]+.3], facecolor='red', alpha=0.3)
   %    if treatment[i]:
   %        pylab.text(X[i,0]-0.2, X[i,1], '%d' % sample['sample1'][i], ha='center', va='center', size=10)
   %    elif not treatment[i]:
   %        pylab.text(X[i,0]+0.2, X[i,1], '%d' % sample['sample2'][i], ha='center', va='center', size=10)
   % pylab.gca().set_xticks([]);    pylab.gca().set_xlim([-1,11])
   % pylab.gca().set_yticks([]);    pylab.gca().set_ylim([-1,21])
   % 


   \begin{frame}
   \frametitle{Box model (for binary outcome)}
   \begin{center}
   \resizebox{!}{2.7in}{\includegraphics{./images/inline/a6f3e6d63b.pdf}}    
   \end{center}
   The entire sample (observed).
   \end{frame}

   %%%%%%%%%%%%%%%%%%%%%%%%%%%%%%%%%%%%%%%%%%%%%%%%%%%%%%%%%%%%

   \begin{frame} \frametitle{Binary response}

   \begin{block}
   {Example (continued)}

   \begin{itemize}
   \item We know
     $$
     \begin{aligned}
     \widehat{p}_{\text{treatment}} &= 82 \% \\
     \text{SE($\widehat{p}_{\text{treatment}}$)} &= \sqrt{\frac{.82 \times .18}{100}} = 3.8\% \\
     \widehat{p}_{\text{standard}} &= 82 \% \\
     \text{SE($\widehat{p}_{\text{standard}}$)} &= \sqrt{\frac{.66 \times .34}{100}} = 4.7\% \\
     \end{aligned}
     $$
   \end{itemize}
   \end{block}
   \end{frame}

   %%%%%%%%%%%%%%%%%%%%%%%%%%%%%%%%%%%%%%%%%%%%%%%%%%%%%%%%%%%%

   \begin{frame} \frametitle{Binary response}

   \begin{block}
   {Example (continued)}

   \begin{itemize}

   \item Therefore,
     $$
     \text{SE($\widehat{p}_{\text{treatment}} - \widehat{p}_{\text{standard}}$)}
      = \sqrt{(3.8\%)^2+(4.7\%)^2} = 6.1\%
     $$
     and
     $$
     z = \frac{82-66}{6.1} = 2.6
     $$
     \item The one-sided $P$-value is .5\%, when we test
     $H_0: {\color{green} {p}_{\text{treatment}}} \geq {\color{red} {p}_{\text{standard}}}.$
   \end{itemize}
   \end{block}
   \end{frame}

   %%%%%%%%%%%%%%%%%%%%%%%%%%%%%%%%%%%%%%%%%%%%%%%%%%%%%%%%%%%%

   \begin{frame} 

   \end{frame}

   \end{document}
